\title{The Libertarian}
\author{Jaroslav Tucek}
\date{}

\newcommand{\sectionline}{
	\nointerlineskip \vspace{\baselineskip}
	\hspace{\fill}\rule{0.15\linewidth}{.7pt}\hspace{\fill}
	\par\nointerlineskip \vspace{\baselineskip}
}

\newcommand{\makecopyright}{
	\clearpage
	\begin{center}
	Text copyright \copyright{} 2015 Jaroslav Tucek

	All Rights Reserved
	\end{center}
	\vfill
	\clearpage
}

\usepackage{graphicx}
\usepackage{epigraph}
\usepackage[utf8]{inputenc}
\usepackage{hyperref}
\usepackage[ngerman,english]{babel}

\hypersetup{
	colorlinks,
	citecolor=black,
	filecolor=black,
	linkcolor=black,
	urlcolor=black
}

\begin{document}

\selectlanguage{english}

\frontmatter

\makecover
\maketitle
\makecopyright
\tableofcontents

\mainmatter

\chapter*{Preface}
\markboth{}{Preface}
\addcontentsline{toc}{chapter}{Preface}

\epigraph{I swear by my life and my love of it that I will never live for the sake of another man, nor ask another man to live for mine.}{John Galt}

It has been more than fifty years since Ayn Rand published her magnum opus. Since that time, governments have grown even larger whilst our liberties have continued to shrink. But no charismatic leader has risen to challenge that trend. No John Galt walks our world and it is doubtful whether he ever could. The collectivist doctrine of the society's supremacy over a man appears to have won in the West. What choices are left open to us, as individuals, to live our lives without becoming enslaved cogs in the machines of the welfare states?

This short essay-story attempts to demonstrate, by presenting the possibility of going on a limited form of a Randian strike in the 21st century, that there is always a choice, and that we are not entirely powerless.

No previous knowledge of Miss Rand's ideas is assumed of the reader, however, no attempt has been made to write a book offering a self-contained argument for their merit, either. That book has already been written; its name is \emph{Atlas Shrugged}. We do not concern ourselves with fully demonstrating the justifiability of the Randian strike, but merely with exploring the plausibility of its execution in today's world.

The second, and perhaps the more important, aim of the story is to show that, while the price paid for the strike is very high, it need not be anywhere as high as portrayed by Rand. Francisco d'Anconia would have to be a patented fool to keep Dagny in the dark about what he was doing the way he did---for twelve long years. This might make for a captivating mystery fiction, but also for a poor character analysis, especially considering that, for Rand, joy is the goal of existence, and sacrifice a dirty word. From that standpoint, the choices made by some of Rand's characters are most extraordinary and bordering on the bizarre, for self-sacrifices they truly are---Francisco fully deserves to lose the woman. At one point, when the two of them meet at Rearden's wedding anniversary, he calls her \emph{a magnificent waste}. In fact, he can just as well be describing himself. The extent of the magnificent waste of human joy carried out by the strikers---not against the looting world, but directly against themselves---is both appalling and unnecessary.

This is a story of two people finding their way toward each other, and toward their freedom. While John Galt and his strikers would have surely considered the rebellion of our protagonists as inadequate, it is the hope of the author that the reader may nevertheless find in this tale if not inspiration, then at least a brief moment of amusement.

For obvious reasons, this book would not have been written had Ayn Rand not written hers. I would like to acknowledge the debt which I owe her, both as an author and as a man.

In addition, I would like to thank Jessica Carter, Peter Grandcourt and Derek McGowan for their careful reading of the manuscript and for their helpful comments. My thanks go also to Francisco Ferrer, for helping steer the main protagonist of the story toward zero-state beliefs, and to the artists at 2nd Right Hand, for creating the book's cover design.

\hfill Jaroslav Tucek

\hfill May, 2015

\chapter{The Gorgon}

\epigraph{Each man kills the thing he loves.}{Oscar Wilde}

This is not the time to feel. About anything. Luca scolded himself at the sight of twin aircraft crossing the sky, their jet engines leaving long white trails of water vapor behind them. The trails were slowly spreading outwards, forming miniature clouds on an otherwise completely cloudless sky. He hastily turned his sight away from the spectacle, struggling to deny to himself the exact nature of the feeling that it had awakened in him. He had a job to do. He'd worry about his feelings later.

The streets of München around him were teeming with life. The warm Saturday drew throngs of locals into the parks and restaurant districts of the city---there to be joined by thousands of tourists already pouring in to enjoy the picturesque Bavarian capital.

Luca strolled down one of the cafés-lined streets of the Innenstadt, watching people all around him. The slow, relaxed movements of the crowds were periodically broken by the measured, purposeful strides of waiters attending to their guests.

The voices filling the air merged into a single hum beyond anyone's ability to distinguish particular words. It carried but one tone now---the boundless joy of being alive on this day. Luca smiled to himself, having finally chased away from his mind the last thought of the troublesome aircraft. This was his second week in the city, but he already knew that he was in love with it.

He had long removed his jacket and was rolling up the sleeves of his white shirt now, eager to feel the sun's rays touch his bare skin, and thinking of the untamed beauty of the Bavarian Alps within his reach, just outside the city.

The sudden frown upon his face was quickly followed by a reproach. Not again! You lose focus too easily, pay attention. You have a job to do. Afterwards, you can run to your precious mountains.

Luca sighed, thinking bitterly about the kind of job he would now have to do. It took all his willpower to resist the urge to look up at the sky again.

The full recognition of the thought darting through his brain hit him with the equivalent of a physical punch. He stood still, petrified, and felt beads of cold sweat forming on the back of his neck. This was the closest he had ever come to quitting.

He pressed the back of his hand to his forehead and took a deep breath, recapturing full control of his thoughts again. Moving with deliberate slowness, he put a hand over the photography bag hanging over his shoulder for reassurance, and felt relieved as total clarity returned to his mind. Then, he resumed watching people in the cafés.

This was his third trip to the city with the same purpose, the two previous ones unsuccessful. By the time he cleared the most popular areas of the city, he was beginning to fear that the third attempt would end equally fruitlessly.

Turning a corner, he entered the cooler streets on the banks of Isar, looking for a quiet place to grab a cup of coffee for himself.

But then he saw her, sitting alone and reading a book in an obscure little place somewhat deprived of a direct view of the river. She wore a plain, unassuming pleated dress, its light blue color blending smoothly into the whiteness of her skin that appeared as if impervious to a suntan. Hundreds of freckles covered every inch of her face, rising from the bottom of her chin all the way up to meet the red of her short hair.

Luca watched her for a minute---watched her sip from a cup of coffee, turn a leaf of her book, unconsciously run a finger through her hair.

Ah, but there is so much beauty even in the city, he thought, a faint smile forming on his lips. Then, he entered the small café and approached the girl's table.

\selectlanguage{ngerman}

``Entschuldigung, meine Dame, sprechen Sie English?''

\selectlanguage{english}

She looked up from her book, a little startled by the stranger's approach and a bit amused by his demeanor. Her eyes were pure green. ``Yes,'' she replied shortly.

Luca smiled and nodded in acknowledgment. ``I hate to interrupt your reading, but would you give me five minutes of your time?''

The girl's amusement remained, but hints of wariness were mixing in. ``Sure, sit down, please,'' she said, placing her book aside with a move of her arm. She couldn't have been more than twenty years old.

``Thank you. What did I interrupt?'' said Luca pointing to the book.

``Oh, that? It's just something we're supposed to read for a university course---on the limits of democracy.''

``Anything interesting?''

``Not really.''

``Isn't it a terrible waste to spend such a day on uninteresting matters?''

The girl remained silent for a while, watching him. ``And what about you, mister? Are you spending your time on matters of ample interest today?''

``Assuredly. What would you say if I told you that you're the most beautiful sight I've encountered today, or in months actually?''

She turned away with something of an air of embarrassment. This wasn't a reaction he had expected. ``I'd find that hard to believe. In addition, I'd have to question your motives in speaking to me.''

Luca was smiling again now, openly. ``If I may dare to suggest, madam, you should spend some time instead investigating the limits of your beliefs. I meant what I said, although the beauty of your freckled face is hard to put into words.''

The girl blushed slightly, but did not look at him. ``I hate my freckles.''

Luca kept studying the girl without speaking. She remained turned away from him, staring into the table. He felt an unfamiliar kind of anger at her words, as if at the thought of loathsome hands grappling something delicate which they had no right even to approach. The indignation filling his awareness kept repeating a single question---is this what she is made to feel for being different? He wanted to reach over the table, seize her shoulders and shake her body, crying---don't give it up to them! It wasn't clear to him who exactly he meant by them, but he was certain that the girl's plight was somehow profoundly tied to his.

All remaining doubt and hesitation then vanished from his mind without a trace---this was the only kind of job left open to him. He looked at the girl and asked with a tone of patient kindness, ``Why?''

``I am not sure. Do you know why you hate or love the things you do?''

``I try to find out in every case. The notion of a causeless love or hate strikes me as something thoroughly absurd. I do not believe that you have thought it through. Why would you hate a gift that sets you apart from the majority of women, placing you into a niche of which you are easily if not the queen herself, then at least very high nobility?''

She looked up from the table, facing him with a confused expression. ``I do not know. It is simply something I cannot help,'' she added meekly.

``That is a pity indeed. I recommend that you do think it through. It is your life, nobody else's.'' He finished with a stern tone, and held her confused gaze for a moment. She looked at him helplessly and made no move to answer.

Luca went on, in a much softer voice, ``What if I told you next that I would like you to pose for me for a photograph? I am prepared to offer you five hundred Euro for the picture.''

Her confusion was changing to incredulity now. ``Why me?''

``For the forlorn beauty of your freckles contrasting against the innocent white of your skin. For the green eyes sparkling with amusement one minute then overflowing with sadness a moment later. For the air of fragility about you that makes me want to protect you with my life, yet somehow half-expect it is me who ought to be protected from you.''

She remained silent, not stirring, only the movements of her dress betraying the faster beating of her heart.

``What is your name?''

``Marlisa.''

``My name is Luca.'' He extended his hand.

She slowly reached out to shake it. ``What kind of a photograph?''

``I am an independent photographer working for a client who requires a photograph of a modern rendition of a Gorgon. Are you familiar with these creatures? They were ancient Greek sea demons. Female beings so dreadful that the very sight of them turned mortal men to stone. I want to turn that concept around and make a photograph of a woman of such haunting beauty that the very sight of her will have the same effect. You'd be in the photograph with a marble sculpture of a hapless man, who's already paid the highest price for the privilege of laying his eyes on you. His face is contorted with the pain of betrayal, yours with the terrible loneliness of facing eternity on your own, lest you destroy that which you love the most.

``I have already found a great sculpture, in the Haus der Kunst. That part was easy. But I had to spend three days looking for you. You are everything I dreamt of for that picture. The contrast of the freckles against the white of your skin is so beautiful that you wouldn't even need any make-up. Your face is perfect as it is. Are you interested?''

She did not answer. She was blushing and made no attempt to hide it.

Luca stood up, offering her his hand. ``Come, the museum is not far from here.''

She grabbed the extended hand and joined him, leaving the café.

\sectionline

\ldots{}lest you destroy that which you love the most, Luca repeated wordlessly walking alone again, no longer able to deny to himself that the feeling weighing down on him was loneliness.

He thought briefly about Marlisa's smiling face, but promptly shook his head with slow, helpless moves---no, she was not the kind of cure he needed. His sight shot up toward the sky, but the aircraft were gone and their trails long dissolved into the unbroken azure of a late afternoon sky.

Let me go, my love. His unspoken appeal to the unnamed presence in his mind was drowned in the most merciless and the least bearable form of pain---not the pain of unrequited love, but of love that must not be requited.

Luca shuddered, feeling suddenly cold despite the warmth of the day, and put his jacket back on. He saw the spires of the Frauenkirche sparkling in the sun, yet for some reason the sight filled him with an ominous sense. He felt a brief moment of kinship with the spires---hopelessly reaching for something that is never to be reached. The city had made him feel so alive just a while ago, but now he felt like a stranger in an alien land.

And rightly so, thought he, for an alien you are. It's not the city who deserted you. It is you who deserted the city. Years ago.

What is it that you still hope for? The life you lead \ldots what has become of it? It is but a race against death now, a race that cannot be won, and every step you take down the road you've chosen like a footstep of doom.

His head was shaking even before he finished the thought. What's the matter with me today? I haven't given up the future. And I never will. He shook his head again, energetically this time, and walked on with renewed resolve, trying to recapture a sense of purpose amidst the desolation stripped clean of hope, meaning and desire. No, this was not the time to feel. About anything.

\sectionline

It was a tiny, dark office filled with stacks of print-outs and permeated by an ever present smell of cigarette smoke. Luca's client was studying the mournful, freckled face gazing at a statue of a contorted man in black marble.

``You are expensive, Herr Segreti, but worth it. I was told, that you can deliver a most remarkable, unique photograph on any subject we can name. I see that reputation is well deserved. We will buy this one. My secretary will help you fill out a contract guaranteeing our payment in two days.''

``Herr Dirchs, do I look like a lawyer to you?'' The self-control of his voice just barely failed to mask the tinge of contempt. ``Here is your photograph, and here is the release form from the model. As to the payment, do I have your word?''

\selectlanguage{ngerman}

``Natürlich.''

\selectlanguage{english}

``That is good enough for me. We'll leave the contracts to the lawyers. Please send a check to this address, pleasure doing business with you.'' He half-rose from his chair and handed a small paper note to the client.

The chief-editor watched him lay the document and the CD on the table, then took the note with an address from his hand. ``Not a funds transfer?''

``No.'' The answer carried a tone of finality.

The chief-editor raised his eyebrows at the unusual request, but then only shrugged his shoulders, it mattered to him not. He liked the young man---quiet, competent, straight-to-the point and with no pretensions about doing anyone any favors. ``Would you be interested in another such deal?''

``Anytime.''

``We would like a short series of photographs on the theme of sin and redemption. Do you think you can provide those?''

``When do you need them?''

``Only for the next issue, in three weeks or so.''

``Sure.''

``Do you give any discounts on your per-photograph rate for a series?''

``No.''

``You are expensive, Herr Segreti.''

``I'll make it worth your while. \begin{otherlanguage}{ngerman}Auf wiedersehen, Herr Dirchs.\end{otherlanguage}''

``\begin{otherlanguage}{ngerman}Auf wiedersehen, Herr Segreti.\end{otherlanguage}''

\chapter{Sin and Redemption}

\epigraph{Justice? You get justice in the next world. In this one you have the law.}{William Gaddis}

He finished his lunch quickly, the meal hadn't been particularly enjoyable. Well, there is one thing the German barbarians would do well to learn from the civilized south, Luca thought to himself only half-jokingly and waved at a passing waiter.

He kept his attention on his laptop, browsing through photographs of various German churches, looking for a particular setting for a scene that was beginning to form in his mind.

The waiter approached, bringing the bill. Luca watched his swift, expert movements and polite manners with enjoyment. They improved his assessment of the diner, somewhat.

He paid for the lunch, folded his laptop closed and rose to leave. He had to sit down again, however, at the sight of a woman entering the diner.

She was dressed in trim office clothes stressing her tall, firm figure and the flawlessness of her posture. Two men in slacks and shirts accompanied her. Probably employees from one of the nearby business centers on a lunch break, assessed Luca. The woman was talking to the man to her left in a lively way, her sparkling eyes laughing more brightly than any smile on her lips ever could. The two men were obviously captivated by their companion and she knew it. Her head held high with a defiant pride as if throwing a dare to the world---try to earn me!

She walked to an an empty table and took a seat. The two men followed her, leaving no doubt about who was the leader of the group. She sat with her back toward Luca. The only thing remaining visible to him was the raven black hair reaching down to her shoulders, there to meet the white of her blouse.

He smiled to himself. Just when he had found his redemption, there was his sin walking up to him on its own. It was unfortunate that she wasn't alone---the risk of rejection was always so much higher in a group---but he had nothing to lose. He waited for the waiter to take their orders, then stood up and walked to the table.

``Excuse me, gentlemen, my lady, can I have five minutes of your time?''

As he had expected, it was the woman who answered, without any change in her smiling expression. ``No problem, take a chair.''

``Thank you. I do not wish to interrupt your conversation here for too long, so let's get straight to the point. I would like to present a business proposition to you.''

``A business proposition for us?'' she coyed, still smiling.

``No, for you alone.''

The smile disappeared, replaced with a look of serious attentiveness. ``What kind of a proposition?''

``My name is Luca Segreti. I am an independent photographer looking for a model for a special assignment on behalf of a major German magazine. I would like to take three or four pictures of you. Are you interested?''

``Pictures of me, Mr. Segreti?'' replied the woman leaning back in her chair, the amusement returning to her face. ``But I am an engineer, not a model.''

An engineer, thought Luca to himself. I'd never have guessed.

``I do assure you, madam, that if you ever tire of your profession, you can make a fine living in mine. Depending on the number of photographs we actually end up using, I am willing to pay up to two thousand Euro for your time. The client's desired theme is sin and redemption. I have already found a perfect place for the shooting, the Elisabethkirche in Marburg. A dark Teutonic church with beams of light streaming in through tall Gothic windows, like the grace of God reaching down to embrace you. You'd pose against that light in a progression of expressions from defiance through uncertainty to remorse and sorrow. As a woman who had sinned, was proud to have sinned, only to discover what price she now has to pay. Wondering if the likes of her can still be redeemed.''

``Do you consider that a fitting motive to have me pose for?'' the woman threw back challengingly.

``I do. I have yet to meet a saint, my lady.''

She glanced at her two companions, then back at Luca with the amusement of a child faced with a new unexpected game, then added with an air of noncommittal, ``I am afraid you've got the wrong woman. I am not an actor and cannot really see myself embodying all the emotions you have in mind.''

Luca smiled with satisfaction. If she was trying to hide her interest, she was failing terribly. ``In that you are certainly mistaken, madam. Every woman is an actor.''

It were the two men who burst out laughing, though Luca could feel annoyance in the laughter, as if directed at an uninvited guest intruding upon their territory.

The woman, too, was smiling, but was not about to let her challenging tone down yet. ``All right, Mr. Segreti, and how do I know that there is any magazine at all, and you are not telling me fairy tales now?''

``You don't. Some things you have to take on trust.''

``Is that how you propose to do business, on trust?''

Luca studied the woman, with a slight desire welling inside of him to slap her face. He had lived his whole life putting the value of his word above that of any contract drafted by the modern lawyer caste, considering his honor a steel chain stronger than any law telling him how to behave. He disliked any implication that his word was not to be trusted.

You're being unreasonable, he scolded himself, abandoning the thought. She has never met you before. It would be foolish of her to trust indiscriminately.

He opened his laptop, displaying a folder with his past work and extended it to the woman. ``This is a sample selection of the models who I have worked with before.''

She browsed through the photographs, visibly impressed, then straightened herself up and faced him. ``Very well, Mr. Segreti, you win. I am interested. But Marburg is quite far from here.''

``You do not fancy a trip?''

``Oh, a trip is fine, but it would have to be on a weekend.''

``Any day of your convenience.''

``All right. This Saturday then?''

He took the laptop back and looked something up briefly. ``The first train for Marburg leaves at 6:51, I'll get tickets for both of us. Can we meet 6:30 at the \begin{otherlanguage}{ngerman}Hauptbahnhof\end{otherlanguage}, by the ticket offices?''

``Sure.''

``Do you own any black clothes?''

``Oh, plenty. I have a nice, long black evening dress or a business suit with pants.''

Luca slowly ran his eyes down the sleek figure sitting in front of him, not making an effort the disguise the appreciating look. ``Wear the suit, please,'' he said courtly and rose to leave.

``Is that all?'' she blurted out with a look of someone having dozens of questions who suddenly reaches the unexpected end of a manual.

Luca held her gaze with the kind of a challenge she had just withdrawn. ``No. What is your name?''

``Elke Tresler.''

``See you on Saturday, Frau Tresler.'' He inclined his head to all three sitting at the table and left the diner.

\sectionline

Hundreds of people were streaming through the \begin{otherlanguage}{ngerman}München Hauptbahnhof\end{otherlanguage}{} even in the early hours of the morning. Elke stood waiting at the corner of the ticket offices feeling a reckless desire for adventure mixed with a tinge of apprehension. She waved it off with an insolent rebelliousness of a schoolgirl discarding her mother's advice and taking a ride from a stranger. She was a schoolgirl no longer, but a woman strong enough to meet any adversary and proud enough of her strength to enjoy the duel.

There was a tap on her shoulder. She turned around and saw Luca stepping in to face her, smiling. ``Good morning, Frau Tresler. I see I was fully justified in leaving you to choose your clothes. No professional costume rental could have improved on your choice. You look fantastic.'' He extended his hand in a greeting.

She looked at him, matching his smile, and shook the offered hand. ``Thank you. You can call me Elke.''

``Luca. A pleasure.'' He touched her back gently with his hand, steering her in the direction of the platforms. ``Let's go.''

\sectionline

The Deutsche Bahn ICE train raced noiselessly northward through Bavaria toward Hessen. Luca did not speak for a long time, gazing through the window at the rolling countryside while the train approached the Danube valley.

``The German high-speed trains are a marvel. It's not just that they are great, it's the arrogant pleasure of you Germans about them being great. As if you laughed in our faces---of course they are, we made them, didn't we? How else did you expect us to make them other than great? It is what I love about this country so much.''

Elke was studying him with interest. His firm, cleanly shaven face stood starkly in contrast with the carefree, disheveled locks of his short brown hair. His semi-formal clothes were expensive but worn---and he wore them casually, like a man who'd feel at equal ease in them in the drawing room and during a prolonged forest trail stroll. He spoke with a controlled, courteous voice, yet his eyes laughed with lighthearted audacity. There was a strange spark in that silent laughter, as if he was amused by a private joke on the world. She could not quite name what he made her feel.

``Is that why you left your country for ours?''

``No, that is something I happened to discover here only by chance, although I am very happy that I did. I left Italy many years ago, for no reason in particular, living in different countries for months or years at a stretch. It was France whom I left for your country.''

``For no reason in particular?'' she asked with a hint of mockery in the question.

``Would an aimless drifting without reason or purpose insult you?''

``If you want to hear the ugly truth, it would. There is nothing more deplorable than the purposeless.''

His features froze with surprise---and with the shock of surprise at being surprised---of a man who is not accustomed to that kind of reaction in himself. He looked at her inquiringly. Have I sorely misjudged the woman?

Elke found his piercing gaze unnerving, but she held it without flinching. At last, he turned away from her and remained silent for a while. She could not tell if he, too, had found her gaze unbearable or if he sought to hide from her a change in his.

``You are wrong about only one thing. Truth is never ugly, no matter how much it may hurt.'' He looked out of the window with a wistful smile that made her certain that whatever her answer made him feel, it was not pain. ``The world is such a large and interesting place, my lady, full of wonder and diversity. Why limit yourself to experiencing only a small part of it?'' He did not turn to her, continuing to look out of the window.

``I do not mean to deny that diversity is pleasant, but do you not consider such constant restless moves to be a waste of a sort? What about steadiness, steadiness of purpose, of staying in a place long enough to call it a home, being loyal to your homeland and tirelessly working for its better future, and the future of the whole society?''

The bitter chuckle startled her. He was eyeing her now with a look that she thought was reserved for drunks crossing one's way muttering incomprehensible obscenities. He turned back to the window without answering. ``You said that you're an engineer. In what field?''

She found his aloof tone and the change of topic without answering her question mildly infuriating, and she had to take a moment to compose herself in order to reply calmly. ``Electrical engineering. We are providing specialist equipment for the chemical industry.''

``Are you good at your job?''

``I am.'' Her reply came without a moment's hesitation.

Luca smiled. ``You are a true daughter of the industrious German nation.''

She could detect no mockery in the statement, it was a plain tribute. ``I do not understand you. You said that you love the greatness in Germany. Why not stay and work to make her even greater?''

``I cannot answer you.''

``Why not?''

``You would not understand.''

``What a cheap excuse. Can you not at least try?''

Luca sighed and turned to face her. ``What would you say if I told you that that very act would not be the loyalty you think, but a grave treason to everything I hold dear?''

``Then I would have to think that you do not understand the meaning of the words you utter. What treason are you talking about?''

``Treason to my deepest love---my love of justice.''

She stared at him in silence, not understanding.

``And treason to the principle I live by.''

He was still looking in her direction, but Elke was certain that, for the moment, he had completely forgotten her presence, and had drifted into some private world of his, which she could not be a part of.

``And what would that be?''

The question seemed to have brought him back to the reality of the train car. His eyes wandered around briefly, as if taking note of unfamiliar surroundings for the first time, then rested calmly on her. What she heard was a voice of pure clarity and assured firmness. ``That no man, inasmuch as he wants to live as a man, can ever consume the unearned, or grant the undeserved.''

``How on Earth would you be betraying that by working here?''

``If you cannot see how, then ask yourself who benefits the most and gives back the least for every specialist equipment that you produce.''

``Good God, that is ridiculous. Are you saying, it is not that you just won't stay and contribute, you refuse to on account of some misguided principle?''

``I told you that you wouldn't understand. I am simply arranging my life so that my work is of as little use to society as possible---without outright resigning myself to subsistence farming somewhere in Inner Mongolia. That would be the rational thing to do, but I am not a brave enough man to do that.''

``Not brave enough?'' She was staring at him with incredulity and horror. ``Are you saying that being a drifting photographer is just some such arrangement for you, not a first choice?''

The wistful smile returned to his face and he met her eyes directly. ``I have always loved photography. But that love is nothing compared with the adoration that I feel when I realize of what little practical use my work is to the future of the whole society.'' He tossed the answer at her in the manner of a quote. He started to turn back to the window, but stopped briefly to add, ``Originally, I did work in the aerospace sector. We used to build great things then.''

She did not dare to question him further, shocked by the intense pain in his last sentence. The peculiar photographer was turning into a mysterious question mark before her eyes. They did not speak again for the rest of the journey.

\sectionline

The tense atmosphere disappeared the moment they stepped off the train. Luca was smiling at the sight of the medieval city dominated by a castle-crowned hilltop. She looked at him and could not help smiling back. There was not a hint of pain in his face, only the joy of a being that is alive in the world and about to act.

``Come, there is a visagiste studio a couple of blocks away from here'', he beckoned her to follow.

``You do not like the way my face looks as it is?'' she asked playfully looking straight at him.

``As a man? It is a most lovely face indeed.'' He answered a shade too solemnly, holding her gaze. Quickly, however, he resumed walking and went on in a lighthearted tone, ``But as a photographer I like my women with as much contrast as I can get.''

``Your women? What would your girlfriend say if she heard you talk this way, Luca?''

``I try not to get entangled in that way.''

``What?'' she gasped. ``Is this another creative arrangement of yours or the ultimate exaltation of diversity at the graveyard of loyalty?'' She immediately regretted her outburst, remembering the bitterness and pain from the train ride. Attempting to somewhat blunt the edge of her question, she quickly added, ``After all, you obviously enjoy the company of women quite a bit. Surely, you're not trying to tell me that you are a monk?''

Apparently undisturbed by the question, Luca kept on walking with an unchanged smile on his face. ``Oh, I love women very much. I also love going to the zoo, and watching the tigers there. I wouldn't dream of bringing one home as a pet though. They may be majestic cats, but high-maintenance at best, and dangerous at worst.''

She burst out laughing without control. ``You are avoiding giving me a straight answer.''

``Perhaps. But you wouldn't laugh like this if my simile were a completely foolish one.''

They went on in silence, Elke's question mark growing larger still.

\sectionline

The studio was empty and Elke was placed in front of the visagiste immediately upon entering. She had not been in a similar place before and was studying the cosmetic paraphernalia displayed all around her with some trepidation, as one may regard the drilling instruments when lying on a dentist's chair.

Luca, on the other hand, appeared quite at home and was explaining excitedly to the visagiste exactly what kind of make-up he had in mind for Elke.

``No, that is not necessary, keep her skin quite plain.'' He put his hand on her shoulder and kept on talking to the visagiste.

She could not think of anything else than the hand, touching her with an arrogant gesture of a man grabbing his property. The words \emph{my women} still rang in her mind.

``Yes, stress the eyes, especially the eyelashes, in blackness to match her hair.''

She wanted to raise her arm and push his aside, reasserting her authority over her body.

``Blood-red lipstick, please. No, something darker. Yes, that is fine.''

She did not move, the annoyance passing into something akin to calm, calm at the sight of a competent man who knows what he's doing and will not be stopped from doing it.

``How do you like yourself?''

The question brought her back from her thoughts, the visagiste's work apparently finished. She looked at her reflection in the mirror. ``It is beautiful. I did not expect it to look like this.''

Luca pulled at her arm, motioning her to rise, then putting both of his hands on her shoulders led her in front of a full-length mirror. ``Yes, you are beautiful.''

She lowered her eyes, but a second later raised them up again with that playful challenge of hers addressing him. ``Luca, you think they will let me inside of a church like this?''

He smiled back with pure mockery in his voice, ``Oh, for the good of society, the Church wouldn't refuse black sheep returning to the flock.''

Not waiting for her answer, he turned to the visagiste. ``Many thanks for your help. I'll pay the bill.''

Walking the streets toward the church, she felt self-conscious for the first time in as long as she could remember, painfully aware of every passing stranger throwing a look her way. As for Luca, he seemed to be completely oblivious to the attention his companion attracted, or else he did not care either way. He wore a solemn expression though and did not smile.

\sectionline

It was dark and cool inside of the Elisabethkirche. Elke rubbed her arms upon entering to ward off an involuntary shiver. A small group of people was praying quietly in the back of the nave. Luca scrutinized the setting attentively, then led Elke to a spot where beams of light poured in to create an airy atmosphere.

``I did not believe it possible for you to look shy,'' he said watching her still self-conscious, standing in the pouring light. ``The woman as of as late as an hour ago was a sparkling challenger throwing a dare to the world---look at me, this is who I am, and I am proud of everything within me that makes me worth looking at.''

She tried to match his stare with a look of injured seriousness, but burst out laughing at the realization of how utterly she had failed to achieve it. She sat down on the floor and faced him with a look of open appreciation of the compliment he had paid her.

``Shhh, this is a house of God.''

She could not tell if he meant it.

``Come now, concentrate. If you have trouble with the pride pose, wait for uncertainty, remorse and sorrow.''

The berating tone was accompanied by a look of unreserved kindness as he approached her and offered a hand to raise her from the floor. She took it and stood up. ``All right, where do you want me to pose?''

The rest of the photo shoot proceeded briskly. She was obediently following Luca's instructions on how to lean and where to look, along with hints on how best to express an emotion, interspersed with technical details here and there.

``Do not look into the camera, look far into the distance to your right.'' ``No, do not look like a child who stole a cookie. Look like a child who stole it and then watched her sister take the blame.'' ``The flash will fire twice, do not move in between.''

She was posing for the last photograph now, kneeling huddled on the floor, gazing longingly toward the beams of light. She felt the cold floor pressing against her legs through the thin fabric of her trousers. Luca's words were reaching her as if from far away, ``You have lived your life like a woman who thought she had abandoned God, only to realize you are in fact a woman whom God had abandoned.''

The cold, the haunting atmosphere of a Gothic church, Luca's words---they all descended upon her bringing her almost to tears. ``Perfect'', she heard from afar and the blinding light of a flash filled her senses. The next thing she knew, there were a pair of hands lifting her from the floor. ``All done, are you cold? Let's go someplace for lunch.''

He was smiling at her packing up his camera. ``And to think you claimed to be no actor. You looked terrific.''

As they were about to leave, they heard a voice from one of the aisles. ``You are so very wrong, mister.'' An older priest was watching them, and obviously had for some time. ``You must realize that God does not abandon sinners.''

Luca turned to the man, his face so uncharacteristically emotionless that Elke was certain, it was maintained only by the greatest of efforts to restrain some violent emotion. ``Father, you must realize what an unspeakable evil you are now pronouncing. If God does not turn his back on the vicious, of what use is virtue? The God you speak of is the worst defrauder of the good that the minds of men can imagine.'' He grabbed Elke's arm and led her quickly out of the church.

\sectionline

They sat in a small restaurant in the historic center of the city. Luca watched Elke voraciously eat her food.

She smiled meekly as a way of apology. ``I have completely forgotten that I had been hungry.''

``I see,'' he took a bite of his own food. ``You have also forgotten to be self-conscious.''

She straightened up, startled, as if suddenly remembering the way she looked, and glanced at the surrounding tables, wondering who else might have been watching her and for how long.

He laughed at her heartily and took another bite of the food.

``Luca,'' she said, leaning against the table to reach closer to him. ``Are you a religious man?''

``Me? What makes you ask that?''

``The way you talked to the priest \ldots and at other times \ldots I simply cannot guess.''

``If by religious you mean whether I believe in God, then I do. But not in the God of that priest, or of any other organized religion on Earth. Their miracles, revealed truths, holy books and dictated codes of morality---fairy tales told to gullible minds in order to keep the priestly circles in power.''

She remained silent, pondering some thought of her own. ``You will find that many Germans are believing Christians, at least officially, but our family has always been very secular. Why do you believe in God at all? What do you need him for?''

``For nothing whatsoever.''

``Then why believe in him?''

``Are you not amazed that there is anything at all, rather than just nothingness? That you will your hand to move and it does? Or at the fascinating regularity of nature, which we can describe with such confidence by the laws of science? To me, God is the reason behind all that.''

She thought about his words for a moment, then went on, shaking her head. ``I do not concern myself with those questions---with the unknowable. The only thing I believe in is reason. My reason. And the only things I am amazed at are the fascinating works that reason enables me to accomplish. Do you understand? Not because there is something greater behind them, but because there is something great in me. And those works are mine. That is my God, if you want to call it that way.''

He listened to her speak with his eyes widening in a surprise of a man hearing a sweet song he has long given up on ever hearing again in his life, ``You are a religious woman.'' It lasted for only a brief moment, however, then he brusquely turned away from her and resumed eating.

Elke had half-expected an annoyed reaction to her words after having attacked his beliefs, and could not comprehend what made him say that. She watched him with a puzzled look, certain of some great struggle going on in his mind, with the sudden attention on his food a futile attempt to hide it from her.

Luca dared not look at her. He was battling the rueful disappointment with himself that it should be his body, not his mind, to be the first to react to her words. The things he wanted to do to her---right then in the diner, in plain sight of the other patrons around them\ldots.

Is that what you want? You meet a rare sight of excellence, like a brief flicker of a star amidst the bleakness of a mooching darkness, and this is all that you can think of? He dismissed the urge with cheerless sobriety.

It wasn't all that he could think of, however. There was also the excruciating pain of seeing clearly what she had revealed to him---the impossible distance between them. She was sitting across the table, but he felt as if they sat worlds apart. A mind like hers, working to what end? He almost shuddered at the thought of how dangerous she was to him \ldots and he to her.

At best, she was a servant of evil. She would not understand. Could he ever hope to explain? And knowing the terrible price of understanding, would he explain even if he could? Would he cause so much pain to that precious mind? It was his loneliness that fueled his desire to make her understand, that much was clear to him. But he must not try to purchase his release at the price of another's misery. If she was happy in her world, he had no right to demolish it for her.

He made up his mind then with dismayed resolve: She must never learn what kind of thoughts I've just entertained. After the photo deal is finished, we'll both go separate ways. It is best that way---for both of us.

Elke was watching him enthralled with fascination. She had no idea what she had done to trigger the tortured struggle that was transpiring in front of her. But the inescapable uneasiness, that was slowly beginning to envelop her, kept forcefully insisting that it was somehow her own fate that hung in the balance.

At last he lifted his head, but his eyes focused far into the distance, looking straight through her, and there was profound sadness in his voice. ``One day, Elke, you may realize what blasphemy you are committing. And to whom you are delivering your God.''

\sectionline

The return journey back to München passed quietly. Luca kept staring out of the window, despite it now being completely dark outside.

Elke could not believe that he would be avoiding her eyes. She sank into her seat and passed the time half lost in thoughts, half fighting the first signs of exhaustion after a long day.

Shortly before the train was about to reach München, Luca broke the silence. ``I will need you to sign a release form for the photographs on Monday, and will also give you your payment. Where can I meet you?''

``What is a release form?''

``A legal technicality. Your written permission for my client to distribute copies of your photographs.''

``I see. Come to my office, say at ten. Does that work for you?''

``It does,'' nodded Luca and noted down the address of the building.

``Will you bring the photographs for me to have a look at?''

``Sure. You can also view them in my online gallery,'' he passed her a business card with the contact information.

``Oh, thank you.''

\sectionline

Elke's office was a small room without windows in a large research and development center. A computer stood on her desk and a number of devices whose purpose Luca did not know lined the walls. There was a single shelf full of books and reference materials. He could not see any personal items except a photograph of Elke with an older couple, her parents, he surmised.

``Do all engineers have their own offices here?''

``No. I told you that I am good at my job.''

``I see your spell of self-consciousness did not leave any permanent damage. All is well then,'' he replied smiling and reached into his bag, putting four printed photographs on the desk.

She studied them all in turn, then looked up to him, eyes sparkling. ``These are amazing. I would not recognize myself. They are beautiful.''

``They are.''

A frown crept onto her face at his words. ``You do not sound very happy about it.''

``I am not.''

``Why not?''

Not moving his eyes away from her, he said slowly, ``There is no sin about you. But that is not what the world will see.''

She watched him, astonished, and not knowing what to say. He simply nodded his head, reached to his bag again, and laid two thousand Euro on the table. ``For your part in the deal.''

He waited until she took the money away, then produced a file from his bag, but stopped abruptly. ``Well, this is embarrassing. I would have sworn that I took the form with me, but the file is empty. There is a printing office in the adjacent complex. I will go print a new form and should be back in five minutes.''

``You can print the form right here,'' offered Elke.

He rose from the desk and put his coat on, ignoring her. ``Is it your employer's understanding that you use his equipment for private purposes?''

She could not believe her ears. ``But it is such a minor thing.'' The sinking feeling within her came in the same instant as the realization that he was staring at her with what was undoubtedly disappointment.

``You see that is the error of most men. They go through their lives defaulting on justice freely, when it's such a minor thing, thinking they will somehow be able to resist the temptation in times of crisis when the stakes are truly high. Or look how they voice righteous indignation at every crooked politician siphoning a couple of millions away, when for the poor guy that is only such a minor thing. How ridiculous is that? Justice is an incorruptible idea that does not allow for degrees of being just. To a drowning man, it does not matter if he drowns in his pool or in the middle of an ocean. You either drown or you do not. You are either just or you are not. I will be back in five minutes.''

Returning to the office, he found her sitting as she was when he left. ``I apologize, Luca, I had not thought it through.''

He nodded without any resentment in his face. ``That is all right. I trust you will think it through now.'' He handed her the release form, indicated which parts she had to fill in, then leaned back in his chair and waited for her to finish.

She returned the completed form, he put it in his bag, smiling, and shook her hand. ``Pleasure doing business, Elke.'' He nodded his head for the last time and turned to go.

This moment was bound to come. She had spent the last half an hour knowing it and hoping that he would prevent it. But he had not. She felt immense heat rising in her chest and her heart beating too fast at the sight of his back. Is he really leaving? She could not identify her reasons and had no time to consider them. Something precious was slipping away from her life and she felt the urgency to stop it at any cost.

As he reached for the doorknob, he heard a gasped out, ``Will I?''

He turned back to see her staring down at the desk, not making any sound for a while. Then, raising her eyes to meet his, she finished the sentence. ``Will I see you again?''

She did not know how much time had passed in silence before he spoke, the only thing she was aware of was the sound of the beating of her heart. She had acted, no matter what his reply would be, she had acted and felt so radiantly alive through acting, life was as it ought to be.

At last he spoke, but there was no smile on his face. His voice carried a hint of defeat and there was poignant disappointment in his glance once more. She could not decide whether he was disappointed with her of with himself.

``I am going to waste the rest of the week being completely unproductive in Berchtesgaden. Let us meet for dinner on Saturday.''

``I'll be happy to,'' she breathed out, not quite able to adjust to his changed tone.

``There is an excellent Greek restaurant at Schmellerstrasse, called Anesis. Do you know the place?''

``No, but I will look it up.''

``I will book a table for Saturday, six o'clock. See you there.''

\sectionline

Elke held her breath as she watched the simulation running on her computer screen. She did not know what time it was, but the vague awareness of being hungry told her that it was probably very late. She dismissed that awareness with a swift, disciplined command---she refused to give herself permission to feel hungry.

Most of her staff had long gone for the day, only one of the engineers remained, apparently reluctant to leave while the team-leader still continued to work. She could feel resentment in his bearings, however, for keeping him up in the laboratory this late.

The room suddenly filled with green light. It looked to her like the light of the brightest morning showering her face in the middle of a spring meadow. It came from the computer screen as the simulation finished, successfully verifying their circuit design. She exhaled in relief and thought back to the months of effort lying behind her. They mattered not now. After months of struggling in vain to design the circuit without exceeding the stringent voltage requirements, they had finally succeeded. It was but a part in their current project of almost two years. But a crucial part. A lot of work still remained to be done, but Elke was certain that nothing could stop them at this point. We've made it. As a payment for every restless night, for every spurt of effort to overcome some unforeseen complication, for every shock of hopelessness when that effort failed. She closed her eyes for a brief moment, allowing the emotion to run freely. I've made it.

She turned her head and smiled elatedly at the engineer sitting next to her but immediately felt the pain of disappointment as he smiled back. She wanted to see a grinning smile of an ally in a long arduous battle saluting her to a glorious victory. Instead, she saw a pleading smile of a prisoner sensing that the moment of his final release is at hand.

``You can go home now, Thomas. I will be leaving too,'' she said faintly and watched him say goodbye and leave the lab.

She wanted to celebrate, but a feeling of loneliness washed over her at the realization that there was not a single person in the whole world who would understand her at this moment and who would be able to join her in that celebration. Even her colleagues were gone. She wondered why it should be in this hour of her triumph and joy that she should feel so lonely. She never felt that way at any other time.

Bright and sociable, Elke had always been popular, yet found it hard to find true friendship and understanding. There had been men in her life before, but none of the relationships lasted over a couple of months. The trouble with you is, you do not have any hobbies, the voice of one of the men kept repeating in her mind.

She threw a sad look at the finished verification displayed on the computer screen and thought of the hundreds of hours of her effort spent bringing it about; of all the little frustrations and then of the moments of joy when things suddenly clicked together and some important problem had been resolved. No, I guess I do not have any hobbies at all, she whispered to herself gloomily.

Had I ever loved those men? No, my work is the only thing I have ever loved. The dispassionate verdict horrified her for a split second, but she could not argue against it. She had never truly loved and never needed another person in her life. It was at moments like these, however, that she cursed herself and desperately wished that she were a woman who did.

She thought of Luca then, confused a little by her desire to run to him at that very moment and show him what she has done---eager to hear that composed voice of his tell her that he saw it and that it was good. But then her loneliness overcame her completely. No, that was the one man who was sure not to grant her this wish. Indeed, something terrifying was telling her that he'd do the exact opposite.

She wondered for a minute why she had asked to see him again. Had that been a mistake? Thinking about her love for her work had filled her with misgivings about their next meeting. Everything rational within her was sure that he would never be able to understand her, but there was this air about him that made her feel as if he were the only one who ever could.

She laughed at herself mirthlessly---you're acting like a fifteen years old again---then powered down the computer and left the room, resignedly thinking of the lonely apartment that awaited her.

\chapter{The Chief-engineer}

\epigraph{Any society that would give up a little liberty to gain a little security will deserve neither and lose both.}{Benjamin Franklin}

She found Luca already seated by the table, his face composed, expressionless. He rose at her approach and bowed slightly in a greeting. ``Good evening.''

His bow suggested an air of breeding \ldots and of detached reluctance. He wore the same kind of expensive, semi-formal clothes she had seen him in the week before, and he wore them as if he did not care. She wondered if she had been correct in her estimate---if he truly wore them during forest strolls and mountain hikes, too.

``Good evening, Luca,'' she returned the greeting smiling, then added, ``Had a pleasant time wasting time?'' She tossed the sentence at him as a teasing remark, but already regretted the biting edge to it.

He did not seem bothered in the slightest though. ``Quite so, I love the mountains.''

He was studying her figure, the composed face breaking into a hint of a smile. ``Is there a special meaning behind your choice of dress tonight?''

She wore an elegant all-black evening gown, the one she had suggested for their trip to Marburg. He had preferred the business-woman's suit then. ``I wanted you to see this. Any regrets about your choice?''

``That is a tough question, but I consider the photographs a complete success. They could have been made differently, not better.''

``Did your client like them?''

``He was delighted. You can also see the popularity of the photographs in the online gallery.''

``I did look at it. You have quite a following.''

``Now you do as well.''

She blushed a little, but quickly continued in her confident tone. ``Why are all your models female? Wouldn't have some of the motives you tried to capture in the photographs been better expressed by men?''

``I have absolutely no interest in photographing men.''

She chuckled, not letting go of her confident, daring tone. ``How do you want me to understand that, Luca?''

``Any way you please.'' He stared right into her eyes, his face still solemn with only a trace of a smile.

``If you wish to understand it in purely practical terms, it would be foolish to over-extend. I enjoy photographing interesting architecture and landscapes, and beautiful women who contrast well with either of those. This is my niche. Let the rest of the market be served by someone else.''

``Are you always like this?''

``Like what?''

``So matter-of-fact about yourself.''

``I am afraid that I have very little capacity for any other type of conversation. You will have to dine with someone else if you are looking for the entertainment of prattle.''

She threw him an injured look but smiled inwardly with satisfaction at his answer. That was decidedly not what she was looking for.

``Fair enough, we can do away with prattle. I must say I am a bit surprised though that you are allowed to display the pictures yourself. Of me \ldots and all the other ones too. Did you not say that they were for publication in a magazine?''

``I do not sell exclusive rights to my work.''

``And your clients are prepared to pay so much money without obtaining exclusive rights?''

``If they do not like my rates, they can always choose one of my competitors. Besides, a promise not to sell rights to any other party seems to always calm them. That gallery is the only other place they will be displayed at.''

``And they just believe such a promise?''

He fixed her with his gaze and did not reply for a moment. ``Some things you have to take on trust, my lady.''

She watched him attentively, some violent desire welling inside of her, foreign and never experienced before. That proud self-assurance, the speech with the strength of convictions of a zealot reciting a long memorized passage, the courteous manner and measured gestures. She wanted to drag him down to her feet, to hear him moan for once and know it was her alone who tore that moan away from him. ``Are you saying that if tomorrow a different magazine approached you to buy the photographs, you'd refuse---no matter what price they offered---even though there is nothing legally preventing you from accepting?''

``Are you trying to insult me, madam, with that question?'' Luca's reply came after a moment's pause, in a grave tone. ``I do not feel obliged to be bound by any considerations for legality. My only concern is for justice. I gave my client my word and I'd sooner boil than I'd go back on that word. Even if abstract concepts like justice or honor do not persuade you, think about the very practical matters at hand. Should I accept the competing bid, I'd earn some immediate income, but by doing so destroy my reputation in the eyes of both my existing client and of the new one as well. I would stand to lose much more in forgone future deals than I could be possibly offered now for a couple of photographs.''

Elke listened to his speech without stirring. No, not a zealot, thought she to herself realizing how much examination had to have preceded those convictions. Another feeling was starting to grow within her at the estimation of who of the two of them was likelier to be broken at the other's feet. It was not an unpleasant feeling. ``I am sorry, insulting you was not my purpose. You are right, of course.''

``I am.''

He seemed to be unwilling to continue down the line of that conversation. He dropped the grave tone, smiled and turned his head away, changing the topic. ``Are you familiar with Greek cuisine?''

``No, this is the first time I have been to a Greek restaurant.''

``I recommend that you try the local pastitsio, it is the chef's specialty.''

``Do you eat here often?''

``I have been here a couple of times before. I have to admit German cooking leaves me with something still to be desired.''

``We are not the world's most famous chefs, I'll give you that,'' Elke was saying laughing. ``Have you lived in Greece before?''

``For a month or so. When I was younger I used to move much more frequently and chose destinations that promised the exotic and the exciting. I lived in many of the Balkan countries and also in northern Africa. But living in an environment where locals carry automatic weapons during routine trips to the marketplace makes one redraw the boundaries of his excitement. I did not stay there for long, returning to Europe. The only countries where I lived for a considerable amount of time were Spain, Portugal and France. And Germany I do not plan on leaving for a while too.''

The waiter came to take their orders.

``I have already expressed my disagreement at the constant moves and I will not repeat it now. But I must concede yours sounds like a very enjoyable lifestyle.''

``What do you enjoy in life, Elke?''

``Me? I guess I do not have much time for that \ldots but I do enjoy my work very much. I really mean it. You may not understand.''

``I do.''

His plain statement struck her like a lightening bolt and for a moment she felt an overwhelming desire to tell him about her circuit board, as if to introduce her own child. But then she saw the wistful look from their train trip return to his face and felt some unbridgeable divide between them. She could not bring herself to speak of it. She studied him for a moment but chose not to question his look, instead deciding to lighten up the air a bit. ``I also enjoyed posing for your photographs a lot.''

``I thought you did.''

``The only other thing I can sincerely say that I enjoy is music.''

``What kind?''

``Oh, I do not have a favorite genre, but it has to be melodic, I do not like jarring sounds. I also do not particularly like purely instrumental music, including classical compositions. I understand that this is something of a mark of the uncultivated, but I cannot help it. That is, unless the classical music is accompanied by a vocalist. I do love those. One of the most pleasant surprises for me lately was the discovery of the music by Núria Rial. Have you ever heard her sing?''

Luca was leaning casually against a side of his chair listening closely to her talk. ``I have attended her performance in Barcelona. She was with the l'Arpeggiata group. That is an angelic voice indeed.''

She smiled at him brightly, pleased by his acknowledgment of a value she too did hold.

Their meal arrived and they started eating.

``You were right, the pastitsio is excellent.''

``There is much excellent food in all the Mediterranean cuisines. Italian may be world famous, but I think Greek and Spanish deserve far more attention than they receive.''

``Pity, Luca?''

``No. It is a loss for the world, not for the Greeks or the Spaniards.''

``You say that as if it pleased you.''

``It pleases me not.''

Elke did not believe a single word of that answer. There was a numb, resigned patience in his voice, yet his eyes betrayed a bittersweet amusement at that kind of loss \ldots and more, they betrayed entertainment by the fact that she had been able to notice it.

``So, what do you enjoy in life? Besides ethnic dining, strolling along the mountain sides and looking for unusual women or places to photograph, of course.''

``You don't think that would suffice to fill one lifetime?''

She did not answer, fully expecting him to answer himself in the negative.

He smiled as if he understood and carried on. ``The only other thing I still enjoy is literature.''

The significance of the word \emph{still} was not lost on her, but again she preferred not to prod. ``What book are you reading right now?''

``I am reading Müller's Atemschaukel.''

``Oh, I did not realize you speak the German language.''

``I don't, not very well yet. The only languages I can say I've really mastered are Italian and English. But I am comfortable enough with Portuguese, Spanish and French to read a contemporary, non-technical novel without a dictionary---or handle my end in a conversation as long as a native speaker is willing to slow down. I fully intend to learn German up to that level too and will stay long enough in the country to do so. Right now, that book is quite difficult though.''

``I see. I will have to re-asses my claim about wasted time a bit, it seems.''

``German is the language of \mbox{Goethe} and \mbox{Schiller}, of \mbox{Nietzsche} and von \mbox{Clausewitz}, of \mbox{Riemann} and Max Born. Had I spent much time in this country and not tried to learn the language of those men, I would have to agree with you and consider my time wasted.''

``You are interested in works of all those men?''

``I am.''

``I do not understand you, Herr Segreti,'' she replied with mock formality.

``You over-complicate things, I am a simple man.''

She burst out laughing with pure amusement. ``That has to be some new meaning of the word simple.'' She resumed eating her meal, smiling. ``What do you have planned for tomorrow?''

``I am going to explore an abandoned tin factory outside of Nürnberg.''

``What? What for?''

``For no reason other than my personal interest.''

``You've taken an interest in tin making?''

``No, an interest in witnessing the abandoned ruins of that which once was greatness.'' Elke stared at him silently, she did not as much as blink. ``Besides, the place might provide some interesting photography opportunities, too. And thinking about it, I haven't been to Nürnberg before either.''

``Do you care for company?''

Luca leaned back, considering her for a minute.

She saw none of the sadness or disappointment from their last meeting, but his reluctance was unmistakable. ``I was just wondering. You do not have to change your plans.''

``Have you seen \emph{The Matrix} movie, Elke?''

``What?'' she exclaimed in bewilderment. ``Sure. But what does that have to do with anything?''

``It is a thousands of years old allegory, going back to Plato's Parable of the Cave. There is a villain in the movie, forgot his name, who betrays the rebels to the agents in exchange for his return to the matrix. What do you think of the man?''

``What is there to think? He is a traitor. And worse, he is a fool---who wishes for the blissful ignorance of a fake reality.''

``Yet, he is a deeply unhappy man.''

``That's tough. Unhappiness is often the price of knowledge.''

He looked at her with the kind of an amused look parents may give a child who talks in a serious tone about theoretical concepts far beyond its possible level of understanding. It was not a happy amusement, but his reluctance was gone. Elke had obviously answered some internal question of his own, but she could not tell what it was.

``Meet me tomorrow at the \begin{otherlanguage}{ngerman}Hauptbahnhof\end{otherlanguage}{} again. Same place, same time as last week. Wear something you would not mind to see ruined.''

\sectionline

They walked through the streets of Nürnberg in silence. Luca was watching the historical monuments, taking a picture from time to time. Elke was watching him. There was an odd intensity about him today, and an air of brooding. She wanted to talk to him, but could not find the words.

At last they came to the imperial castle and stopped by an opening which offered a view of the city below. Elke leaned against a low wall to enjoy the scenery. When she turned back, she found him looking at her. ``Are these the kind of clothes you would not mind to see ruined?''

It was an unexpectedly cold morning and Elke had put on a plain woolen sweater and a pair of her older pants. ``I do not own any garments that are exactly throw-away. But if it comes to that, I'd miss these ones the least.''

He nodded, but did not reply. She would not, however, let the opportunity to break the silence pass. ``What is troubling you, Luca?''

``Why do you think there is anything troubling me?''

``I just know. Trust me, it is a woman thing.''

``You are avoiding giving me a straight answer.''

``Well, so are you.''

She was startled by his laughter. It belonged to a carefree man---openly, joyously admitting he had lost in a match with a worthy opponent. Yet Elke had no doubts that he was anything but carefree.

Slowly, he stopped laughing and walked up to her, a sober look returning to his face. He put a hand on her shoulder and turned her around to face the city, sweeping over it with his other arm. ``Look at this, Elke. The city of \mbox{Frederick II} and Albrecht Dürer, and then of \ldots of Hitler and Goebbels. Doesn't it make you sad?''

She was trying to grasp what he really meant by the speech, but he did not wait for her to reply. ``Come, there is one place I really wanted to see here today. Let us go to the Nazi party rally grounds.''

\sectionline

Luca's pensive mood returned as they entered the immense area of the former rally grounds. Elke had never been to the place before either and she, too, was moved by the sinister architectonic megalomania of the Third Reich and by the documentation center's exhibition on the horrors of its reign of terror.

She felt relieved when they finally left the exhibition, as if the air itself suddenly got clearer. It was about noon.

``Let us get out of this place. Are you hungry?''

She nodded eagerly in affirmation.

\sectionline

They bought Nürnberg bratwurst from the first street vendor they found and sat down on a bench in a small quiet park to eat.

She wondered at the speed with which his frame of mind could change. There was no trace of that pensive gloom about him now and they chatted without effort. He was telling her about the factory while studying a map for directions. Right now, however, Elke wasn't interested in the factory. ``Luca, why do the rally grounds disturb you that much?''

He looked up from his map toward her, taken aback by the question. ``Do they disturb you not?''

``I mean, what the Nazis did was terrible, but it all happened so long ago, even my parents are not old enough to have lived through it. It is like something from a different world.''

``It disturbs me, because I think the exact opposite.''

``And what is that? Are you trying to compare Germany today with that era?''

``I am.''

His reply left her at a loss for words and it took her a moment to recover from the shock. ``Luca, let me give you a piece of advice for your stay in the country. The war can still be a touchy topic for many people. You are not going to make any friends here with claims like that.''

``I have enough friends. But do not worry, I am not a man given to forcing his opinions on others unasked for. We do not have to discuss this.''

``No, that is not what I meant. I want to discuss this. Have you not just seen the exhibition? Then look around yourself and tell me with a straight face that we are now in an even remotely similar situation.''

``It is a difference of degree only. The rhetoric has changed, the warmongering is gone, the threat of physical destruction has become more subtle, but I think the essence has remained largely unchanged.''

``A threat?''

``Why do you pay your taxes, Elke?''

``Why, because it's the price of living in a civilized society.''

``Really? I do not believe you.''

``Well, you can choose to believe whatever you wish, but it is true.''

He looked at her studiously, crumpling the paper tray from his meal. ``I pay the taxes because I have a gun to my head. The moment the gun disappeared, I'd sure as hell stop paying them. And I think you would, too.''

``What gun, Luca? There are no guns in Germany today.''

``If you think so, try to stop paying and see what happens. Sooner or later, a couple of armed men will appear to explain what an unwise decision that has been on your part. Sure, they'd probably not shoot you these days. Indeed, they need you to feed on in order to survive themselves. But you'd end up in jail. I do not think it's such a terrific improvement.''

``That's outlandish. Those taxes pay for a lot of necessary services that have been asked for by the society through elections---not for raising an army to conquer the world to satisfy one man's hunger for power. We are living in a free, open, democratic society---no one can be physically destroyed by a whim of some dictator. The will of the people is a safeguard against that.''

``The will of the people?'' he repeated with bitterness. ``How many mass murderers among the world leaders have used that phrase? It is a basic psychological weapon to disarm a victim and discourage him from protesting or defending himself no matter what injustice may be perpetrated against him---because it's the will of the people. And these days, a democratically determined will of the people, too. But if three sheep and seven wolves democratically vote on what's for dinner today, what comfort is that to the sheep?''

``You have a quick wit but are abusing it here and you are being disingenuous with me. As I said, there are no wolves in Germany today.''

``I agree, we are still a step above the Nazis. Let me rephrase the comparison then. Imagine there is an elephant and a hundred leeches voting what's for dinner today.''

She laughed out in amusement despite her anger at his fantastic arguments. ``Luca, you want to play games with me, and I can oblige you. I am a strong elephant and I can carry hundreds of leeches on my back without even noticing. If that is the price for my life in society, then I will pay it.''

He did not smile at her words. ``And how long do you suppose you are going to be able to do so? Enough leeches will bring down an elephant just as surely as a wolf will a sheep. You boast of your ability to carry the leeches as if it were something to be proud of, and I admire the sentiment. But in effect, you are condoning of a state where some are held at gunpoint to provide for others. But your crime goes further still. For not only are you feeding yourself to a society of cannibals, you are jovially hammering out cutlery for the feast---so that the banquet may continue once your drained corpse is discarded and a new victim found. Just as you cannot deal with degrees of justice, you cannot deal with degrees of murder.

``Yours is a crime born of innocence and virtue, but that does not make it any less serious. You accept the price you are paying now, but where does it stop? Look what the elephant, that you are, already gives. Think of every electrical equipment that you give gladly, proudly, to the world. It is not enough. Think of the effort, the exertion, the dashed hope at every false start and then of the faceless tax collector coming to take a slice the moment that you are able to feel the exultation of success. You give that as well. But as the leeches multiply and the slices get larger, will you keep giving? Where will that take you in the end? You cannot make terms with evil---once you've conceded that your values are not yours by right, you will lose them all one day. It is your life that is at stake. Or will you give them \emph{that}, too? There is no middle ground between freedom and force, Elke. I think you ought to take better heed of what the rally grounds truly represent. It is both your past and your future. And your strength in carrying leeches will pave the way for it.''

She sat very still, staring at him. He had spoken with intensity and urgency, as if fighting a losing fight to hold back the words which he longed to have her hear. It gave him such an air of fervor and vitality that she could not help but think of his body, and of the thin cloth of his shirt fluttering in the wind. She wrinkled her brow---at herself, but he would not know---and retorted contemptuously, ``You are an anarchist.''

``Why, yes, that's what I am,'' he appeared surprised by her reaction, but not displeased. ``I do not like the word, however, for its implication that a state-less society is inevitably doomed to sink into chaos. I very much prefer the term libertarian.''

``Let's not mince words. It is civilization that the government, and the taxes used to pay for it, has brought us. And it is chaos and lawlessness that the government's disappearance would bring in its stead.''

``You do not seem to place much faith in reason. Why should rational individuals be unable to get along without some bureaucrat with a stick watching? There is nothing magical about that stick. It does not cease to amaze me that intelligent people would question neither the justice, nor the necessity of having a politician caste coercively rule over them. The government cannot create values out of thin air, including order. It can only seize them once they have been created by private individuals in the free market. The law and order the government brings you? Fairy tales told, in government-run schools, to gullible minds in order to keep the politicians in power. And make no mistake about it, Elke, power is the only thing they are after. For they are the worst of men. A man of self-respect and ability has better things to do with his life than govern the lives of others.''

``Gullible minds?'' she was surprised how much the words had stung her, coming from him. ``Do you want to hear what I think about your attitude?''

``On the day that you understand that I do not care about your, the society's, or anyone else's estimate of my attitude, then you will have understood that attitude. I stand behind that claim. There are only four types of men, besides bureaucrats and politicians, who accept the government as useful: those glad to be absolved of self-responsibility for their lives, those fearful of making independent decisions, those thoughtlessly taking things as given because that's how things have always been, and those trying to muscle in on the property of their betters. All four groups are quite happy to hide their true reasons by accepting the government propaganda about building civilization and serving the common good. Which one of them do \emph{you} belong to, Elke?''

There was no malice in his words and he had said them evenly, without raising his voice. It had made her struggle to stay calm a little easier. ``Has anyone ever told you that you are an insufferable man?''

He laughed at her with that disarming air of his, of a man without a single care in the world, ``Has anyone ever told you that you are insufferably beautiful when you shake with anger?''

``Don't change the topic,'' she snapped at him, but noticed that she had stopped shaking. Instead of struggling to keep calm, she would now have to struggle not to smile at him. ``You are thoroughly persuaded that you are right?''

``I am.''

``You do not admit any possibility that you may be wrong?''

``Of course I do. The nature of reality cannot be cheated and she will be the final judge. If I am wrong, I will learn or bear the consequences of being wrong---as justice demands. If you are wrong, you will.''

Her anger was gone, replaced by an insistent need for a careful consideration of his words. She crumpled the paper tray from her meal in the most conspicuous way possible as a substitute for the defiance she was now unable to give off. ``Yes, Luca, one of us will have to learn. Have you learned nothing about the value of law and order in Africa, whom you were so quick to leave for Europe again?''

``Africa suffers from a host of problems, I'll give you that, but a lack of government is not one of them. There is only one objective law---and the government is the first thug to break it---that no man may initiate the use of force against other men. There is only one type of a man able to deal with others without resorting to force, a trader on the free market---exchanging values for values with recourse to nothing else than the power of reason, his and other men's alike. And he must be a trader in all things---material, emotional and intellectual---who will neither accept, nor ask for, unearned goods, undeserved love or unmerited recognition.

``I do not see why the free, unregulated market operating under this one law could not supply order---with private arbitration agencies, insurance companies and security contractors supplanting the government's courts, police and armies to bring retaliatory force against anyone who'd try to get away with its initiation. As things stand now, which laws do you want me to abide by, Elke? The laws that chain me and force me to feed my own destroyers, or the laws that allow me, in turn, to chain and rob my disarmed neighbors? The laws of sacrifice or the laws of plunder?''

The lines of his face hardened and his eyes seemed to both dim with sadness and, at the same time, brighten up with a promise of violence, as he looked back in the direction of the rally grounds. ``In my mind, the idea of having to choose between the government and chaos is laughable. You are right, however, that I am enough of an anarchist---with all the connotations of the term---to consider even the chaos of a law-less society preferable to a system that is propped up only by the silent, thankless toil of its enslaved victims.''

``I am not a victim,'' she snapped, but realized in shock that she felt no anger at the words. For a moment, they looked at each other in silence---she with bewilderment, he with with mocking amusement. Then she shook her head, more at herself rather than at him, sighed and went on quietly, almost sadly. ``You should really choose your words more carefully, Luca. You will not understand this, but I will tell you anyway. I do not toil to prop anything up. If that is a side-effect of my work, then great. But it is the work itself that is both my purpose and my reward.''

``Is it?'' His question had a peculiar, soft quality; his face admonishing hardness. ``Oh, I have chosen my words carefully, Elke. Ask yourself sometime, what kind of rewards you are willing to have the world heap on you---and in exchange for what kind of work.''

\sectionline

They found the road leading to the factory easily enough. It was full of patches of broken asphalt, overgrowing with weeds.

Luca dropped the map into his bag and retrieving a cylindrical object, tucked it into his pocket.

``What is that?''

``A telescopic baton.'' He looked at her calmly. ``You never know how abandoned an abandoned building truly is.''

``You forgot to mention this part of the trip yesterday.''

He threw her a quick glance but saw from the teasing eyes that she wasn't upset. ``Would you have declined to go?''

``I guess I wouldn't.''

``I do not really expect any danger. It is not the common thugs that I am worried about.''

``You do feel really strongly about the uncommon ones though, don't you?''

He looked at her with mocking admiration of her talent for understatements. ``I do. There are no batons big enough for those.''

\sectionline

Nature was rapidly proceeding to reclaim the old factory lot. There were bushes and small trees growing over the entire complex; the fence surrounding it, however, was largely intact. They circled around until they reached a partially collapsed section. Luca climbed up and turned back, intending to offer Elke a hand to help her get over. By the time he turned, however, she was already on her way up the crumbling wall. She deftly cleared the top and dropped down on the other side. Luca raised his eyebrows in surprise, then climbed down to join her in the factory lot.

``I am not a manicured kind of a fancy gal,'' she replied in an answer to his silent, inquiring glance.

``Never crossed my mind.'' His smile of respect underscored the blatant lie---openly meant to be seen through.

They entered the building through an unhinged door of what appeared to once have been an electrical substation, then passed through into the main hall. Grime covered machinery remnants and rusting pipelines filled both sides, leaving a broad corridor in between. They walked along in silence, the beam of Luca's flashlight piercing the dimness of the hall.

Elke listened to the falls of her steps echoing in the darkness and watched the disfigured shapes of what used to be precision machinery once. She wasn't sure whether she was desecrating a tomb or paying homage to some fallen hero of hers, but she had no doubt that she was walking down a graveyard to something deeply humane---the graveyard to purpose and thought.

``I did not think the place would be so eerily \ldots beautiful,'' she whispered, afraid to raise her voice out of a sense of reverence.

She looked at him, and even though she could only discern a dim outline of his face, she knew that he felt the same way she did. The divide between them suddenly appeared insignificant. If he feels this way, and about these things, there can be no divide between us. She wanted to stop him, take his hand and let all, that she was so sure they both felt at that moment, be said between them openly. She felt her hand start to move toward his but then freeze in reluctant suspension. Rapidly, she pulled her hand back, angry at her body's refusal to obey her command; and angry still more at him---for not having done what she wanted him to do, and which she had failed to do herself.

The sudden appearance of bright light ended her reflections. They reached what they had thought was the end of the hall only to discover that the building was L-shaped. The hall turned left and went on for maybe another fifty meters. There was a massive skylight along most of its remaining length.

Luca switched off his flashlight, they turned the corner and went on.

``Look!'' Elke was pointing ahead, to well preserved remnants of power station turbines lying, like two whale carcasses, at the very end of the hall.

She started to half-run toward them, but was stopped by a tug at her elbow so strong it hurt. Turning around she saw Luca grasping her arm. He let go once he noticed her discomfort.

``Be careful, you little fool. This place is a ruin. Always check what you are walking on, or under.'' With a motion of his head he gestured her to look up toward the fenestration. ``One of the glass panes falls when you're there and what then? Let's walk around.''

She looked at the many holes in the skylight, then at the shattered glass below and shuddered. He was right. She looked at him bashfully unable to say a word. He simply nodded and led her around the skylight toward the turbines, not noticing the warmth his protectiveness filled her face with.

She climbed around the turbines with curiosity, like a hyper-active child on a jungle gym.

``I've never seen one like these two before,'' she said gaily leaving the machinery to join him again. ``Why are you laughing at me?''

``You did manage to ruin your sweater after all.''

``Oh,'' she looked at the rust and soot covering her sleeves, but went on indifferently, ``I said I wouldn't miss it.''

``You seem to have taken quite an interest in the equipment.''

``Isn't it fascinating? The first project I worked on after graduating was for a turbine control system. You always get a soft spot for the firsts in life.''

He listened with half-closed eyes and a smile on his lips. Elke thought to herself that this was the first time he looked truly happy---and the radiance in that look was unmistakably hope. After a while he opened his eyes full wide and asked quietly, ``Would you pose for me with the turbines?''

``Anytime,'' she answered, grinning.

He knelt down to a pile of rubble nearby and carefully lifted an object from the floor. It was a factory worker's helmet. Holding the helmet by his fingertips, he put it on Elke's head. ``Try not to disturb the dust on it.''

``What will the photograph represent?''

His solemnity was returning as he looked her up and down. He threw one more glance at the rusted machinery all around them, then answered grimly, ``The chief-engineer.''

She chuckled softly and walked toward the turbines. ``How do you want me to look?''

``Forlorn.''

\sectionline

Sitting on the ICE train carrying them effortlessly back to München, Elke felt the first signs of fatigue setting in, but she brushed them off, feeling vibrantly alive. She had enjoyed this day's trip very much.

``Do you try to sell impromptu photographs, like the one we did today?''

``No, but sometimes I get approached by an interested buyer. Maybe a third or so of them eventually get sold. Do not worry, I would let you know in that eventuality and you'd get your part of the payment.''

``That is not why I asked. Incidentally, is it standard practice in the business for a photographer to pay volunteer models?''

``It is not, but I always pay my way. You can look at it as if I did not pay for your pose, but for your signature on the release form.''

You always pay your way? she thought to herself, the words \emph{trader in all things} ringing in her mind. How would you pay for\ldots. She did not finish the thought. ``How long does it take you to process a photograph?'' she asked aloud.

``Not long, unless I made a mistake during the shooting and badly need to save the picture.''

``I would like to see still today how the turbines photograph comes out.''

``You know, patience is a virtue,'' he quipped.

``Is that a no?''

``No, you can come over if you wish and we'll have a look at it.''

``All right,'' she said breathlessly.

\sectionline

Luca's apartment was a tiny, nineteenth century rooftop studio in Neuhausen.

``My God, you have been burgled,'' exclaimed Elke as she walked in.

``No, it has always looked like this.''

She gazed around with a puzzled look. There was no furniture in the room save for a large, solid looking table with a computer on top of it, and a pair of chairs. The walls were bare, cleanly painted in white. Elke could see that one of the walls hid an inbuilt wardrobe. A broad french window revealed a small terrace beyond.

``Where are your things?''

``I don't have any. That is, besides some clothes in the wardrobe, my camera, the computer and a couple of other minor items.'' He saw the look of disbelief on her face and went on as a way of explanation, ``I could tell you how free and mobile it makes you or how much of your life you save on house cleaning and maintenance, but I won't. That's just accidental. The truth for me is, I love the minimalist aesthetics, the light, the open space, the lack of clutter. Other people's homes always make me feel as if I were in a warehouse.''

She laughed merrily, still looking around, but immediately experienced a sense of overriding guilt for having done so. These were not the things to laugh about. She instinctively felt the same way here that she always did in her laboratory, and could not bring herself to think of the apartment as of a home of a man. It was a dwelling of an impersonal entity, who had---with ice-cold precision---discarded the purposeless, the irrelevant and the inconsequential.

``Or do you miss anything here?'' finished Luca, dragging her out of her thoughts.

She saw a man of flesh and bone standing in front of her and could not reconcile that sight with the image in her mind. She stretched out her arms in a helpless gesture of not knowing where to start. ``Where do you sleep?''

``On a Japanese futon. It is folded away in the wardrobe during the day.''

``What is a futon?''

``A kind of a padded mattress.'' He walked to the wardrobe and spread the futon on the floor.

``You are a monk after all,'' she laughed. ``You can as well sleep on the floor.''

``Why? It's quite soft and comfortable.''

``That's pretty hard to believe from where I stand.''

``Well, you can try it for yourself.''

She paused, a look of seriousness replacing her laughter. ``You do not think I am that kind of a woman, do you?''

``What kind of a woman?''

``That jumps into men's beds so easily.''

``This is no bed.''

She laughed again, admitting defeat, then dropped to the floor and stretched herself out on the futon.

That kind of a woman, repeated Luca to himself. I cannot say I haven't thought of it. What would you look like if I\ldots. But another voice was shouting in his mind. You no longer even have the right to look at her like that. In her world, you have no currency. And in your world, she's got even less. She is the closest to an enemy that you've ever had.

``Oh, looks like you did not lie after all,'' said Elke, quickly standing up again.

Enemies. The word filled his mind with helpless desperation. We, who ought to be comrades in this battle neither of us should have to fight. It was the intensity of the helplessness that told him what the battle was truly costing him, and how much he longed for an ally in his fight.

``Luca?''

The hesitation in her question was like a slap waking him out of slumber. He realized then that he had stood still, silently lost in thoughts, for longer than he intended, and that she was watching him with a perplexed look. He scrambled to remember the last sentence, that he had heard her say, and rushed to reply. ``Of course not. You may think I am crazy, but chances are that one hundred million Japanese are not all crazy. Come, let's go check today's photographs.''

He hurriedly turned away from her, hoping that she did not recognize what kind of thoughts had brought about the look on his face.

He walked back to retrieve his camera's memory card and motioned Elke to follow him to the computer, pulling in the second chair for her to sit on.

She saw Atemschaukel along with an Italian-German dictionary on the desk. Leaves of paper filled with drawings and mathematical equations lay next to them. Elke picked them up for a closer look. He snatched them away from her hands, opened a drawer and let them drop inside.

``What is that?''

``It's nothing, just a bunch of scribbles.''

``If it is nothing, why be so defensive about it?''

He turned away from the computer to face her, sighing. ``It is a design draft of a modification to a variable incidence aircraft wing.''

``Oh,'' she said eagerly, throwing one more glance toward the closed drawer, ``Are you going to seek employment in aerospace once more? There are excellent opportunities for that in Germany.''

``No, I am not going to.''

He turned back to the computer and uploaded the photographs. Continuing to feel her questioning look, however, he added, ``The design was done for my personal pleasure alone and for nothing else.''

She desperately wanted to talk to him now, to understand what pain he was fighting or \ldots inflicting on himself. But she could not find the words and the opportunity slipped away, leaving the riddle of the man sitting next to her unanswered.

He was busy adjusting the photograph and before she could get her bearings, he was tilting the monitor for her to look at, interrupting her thoughts. She gasped at the sight of the picture, involuntarily grabbing his forearm. ``Luca, this is amazing.''

``And completely preposterous,'' she went on with laughter, looking at the stern face of a girl staring back at her from under a dusty helmet, standing with grime-smeared sleeves in front of a husk of a rusty turbine. The words \emph{chief-engineer} raced through her mind in an odd contrast with the vulnerable fragility of what her woolen sweater revealed to be a woman's body.

Dropping the laughter, she spoke to him with quiet seriousness, ``You know, this is who I used to imagine that I'd love to be like. Once, when I was a little girl.''

``You made it,'' replied Luca matching her serious tone. ``From a girl who dreamt of playing with machinery to a woman who builds it.''

Elke watched him with a feeling of her world dissolving in something that she was no longer sure whether to fight or embrace. He understands! And she understood, too, then---what the feeling he always awakened in her was---it was a feeling of certainty. This was not a man of I think, but a man of I know. He did not give opinions, he gave statements of fact. His was not a world of the maybe, but a world of the absolute.

She lowered her eyes, feeling so fully content, yet nervous with anticipation. Her mind suddenly filled with a single thought, of his body just inches away from hers. She did not know how much time had passed like that, a minute or an hour seemed equally likely to her. She was only aware of the way it was broken. By Luca rising from his chair, his words, ``It is getting late, Elke. I will call you a taxi,'' sending a sharp pang of disappointment through her spine.

\chapter{The Mermaid}

\epigraph{We have a system that increasingly taxes work and subsidizes nonwork.}{Milton Friedman}

She'd see him frequently in the weeks that followed, meeting for a lunch during the week and making trips across Germany on the weekends. She would inquire about his travels, he'd ask tirelessly about the country and her life in it so far. She was impressed and humbled by his knowledge of German history, having watched him on several occasions forget himself in lengthy conversations with tour guides at historical sites.

``Did you take such interest in all the countries you've stayed in, Luca?''

``Of course, I am a student of human culture and history.''

``And how does Germany compare?''

``Oh, you know I very much admire the achievements of your people.''

``But?''

``What makes you think there is a but?''

``Because you speak of admiration as if delivering a eulogy.''

He chuckled to himself quietly. ``That I am. I admire you the same way I'd admire a pack of elephants about to wade into a filthy stream.''

\sectionline

Elke had gotten used to bitter remarks like that from him. At first, she had taken them for over-reactions to some unfortunate calamity from his past, to be endured for a while until time makes him forget and move on. But there was an incident in her laboratory last week that filled her with dread now, when she saw his face. It was the dread of doubt. Doubt that it was not her world that made sense, but his. And that it was right for both of them to live in just such a world.

Looking at his face and listening to that calm voice always filled her with a feeling of certainty. But it was not a soothing certainty anymore. It was drowning her in powerlessness and panic.

The incident was an inspection from the European Agency for Safety and Health at Work.

``Do you understand, Luca?'' she had asked when they met for dinner that evening. ``They just came in. Unannounced. And demanded that we stop our work for half a day to let them inspect the equipment. Our equipment!''

``At this time, when we cannot afford to spare an hour, we had to spare half a day!''

``Unsafe levels of electromagnetic radiation! What nonsense. At those frequencies, the radiation is absolutely harmless. Now we'll have to enlarge the lab and cart off a third of our tools to a specially shielded room. It is going to cause a major delay for us.''

``And they will be sending regular inspections from now on, too. You should have seen the kind of equipment they had, though. Wonderful stuff. And definitely not cheap.''

``We got fined---as social justice. They've made it impossible for us to work and call that justice!''

She stopped and looked at him. He sat in his chair, motionless, with a calm, empty expression. She realized that it was her, who had been talking the whole time. He had not said a word.

``I apologize. I do not usually get angry like this.''

He glanced at her with cheerless sympathy, but with almost a wink that made her feel that he was secretly pleased with her anger. His words came out flat, without inflection.

``Whose work has now made it impossible for you to work, Elke?''

``What do you mean?''

``Who has paid for the inspectors and their wonderful measuring equipment?''

``Well, I don't see what you are---'' She stopped, afraid, and faced him blankly.

``They are civilized cannibals and would not eat with their bare hands. Where did their cutlery come from? They don't strike me like the sort of men able to produce it themselves.''

She shook her head uncomfortably, chasing the memory of that conversation away.

Even if Luca were right, how could she ever accept the conclusions he had drawn? What choice is left to the elephants other than to wade in? Should we all try to shake leeches from our backs by becoming photographers? Or Mongolian farmers? We'd all starve, elephant and leech alike.

No, Luca is wrong in that. I will carry my load, no matter how unjust, as long as I have any strength in me left to carry it.

And Luca will come around eventually. Had he not abandoned his desire for excitement before? He will similarly abandon the desire for rebellion. For there were the drawings-covered papers on his desk---if his passion for that is still alive, he will come around. A man cannot turn his back on his love and survive for long. I will be there for him and help him find the way.

I will be there for him, but does he want me to?

She enjoyed their time together, and as far as she was able to tell, he enjoyed her company, too. But in times of their greatest closeness, when she stood barely breathing, her heart racing with anticipation, he always seemed to grow respectfully distant.

Once, not having met Luca for several days, Elke checked his portfolio site only to discover a round faced blonde model on his latest photograph, posing for him in the Goldstedter Moor birch forest. She felt a painful spasm run through her body, which she refused to name to herself. When questioned about it during their next meeting, he simply said, ``You wouldn't have fitted into the scene that well, Elke.''

She forced herself to nod, fighting the despondency taking hold of her. Is this who I am to him, too? Just some model who happens to fit a couple of scenes?

She did not admit to herself, however, how much she wanted to hear him ask her to pose for him again. Until a week later, he did.

\sectionline

``This will be a nude photograph, Elke.''

She looked at him startled but quickly recovered enough to retort. ``Now, now, how would you like to see me naked?''

``If you want to hear the ugly truth, my lady, I don't. Women look their best when dressed well, with the full allure of the hinted, yet forbidden. Put yourself into a tailored blouse and a tight office skirt, lean above a photocopier somewhere, and I'd really love to see that. But undress a woman, and you've stripped all the mystery away. No one wonders anymore, what she looks like underneath. Nudity is cheap. And boring.''

Elke was listening in bewilderment, she did not know whether to blush, feel offended or complimented, and Luca wasn't going to give her time to decide. ``That is, unless you can somehow return subtlety to her, bringing back the element of the hinted.''

``Luca, do you \ldots do you want the world to see me \ldots like that?'' she stammered with resignation. She was sure now that he had always thought of her only as of a model. No man could possibly make this request to a woman who meant more to him.

``Absolutely.''

``For any bum to see me \ldots and to\ldots.'' She did not finish.

He looked at her with what she thought must have been compassion, but with an odd intensity to it. ``Why do you worry about the bums, Elke? Let them see you. Let them see you in all your radiant beauty and courage. Let them see what wonder exists in the world they had abandoned. Let them feel the baseness to which they had fallen, and know the greatness that is now and forever beyond their reach. Let that knowledge burn inside their minds like what it ought to be---the mark of the damned. And to those among them who have not yet lost the last remnants of their humanity, let the sight of you be like a promise and a magnet drawing them up from the depths to which they sank.

``But neither of these should be of any concern to us. Think instead of all the good, honorable people trudging through a thick grayness of their days with an unjust load they could not name, understand or shake off. Let the sight of you be a reminder to them of why they continue to trudge. So that, when at the end of their day, no matter what bitter weariness or desperation weighs on their minds, they could look at you and be glad. It was a day of their lives, and they are glad to have lived it.''

Elke's eyes focused inwards, her head was spinning. She was unsure about the conclusions she drew only two minutes ago. Was he right? Had she not been living like this her whole life? Laying the most intimate part of her, her mind, for the world to see---declaring proudly to all good people able to understand, who she was, and what marvels she could offer them? The bums who could not, or would not, understand be damned! Let them see and use the creations of her mind. That could never detract from what those creations meant to her. He was right. And if she exposed her mind like that, why not her body? She was proud of that, too.

Shining through the fog of her contemplation were rays of light touching her skin with gentle warmth---the way he talked about her, would he talk this way about just any model of his?

He saw the expression on her face but misunderstood it for uncertainty. ``Would you like to take a couple of days to think about it?'' he asked with a touch of tenderness.

``No, I \ldots please tell me about the photograph, Luca.''

``You are to represent a mermaid. I've found the perfect location for the shooting---a hot spring in the Krafta region of Iceland. A stunning landscape. It looks like the surface of Mars. The spring is colored quite opaque and thin vapor rises from the water. Your lower body will not be visible, sitting submerged at the edge of the spring, with your torso resting against the Mars-like rocks---bent to stress the outline of your breasts---and your head lying limply to the side, as if spent, with a look of longing.''

She listened, imagining the scene. ``Isn't it dangerous to enter a hot spring?'' The question belonged to a competent engineer approaching a technical problem.

``Only the volcanic ones, and you can easily tell those apart, the water will be close to boiling. This is a geothermal spring.''

She noted with satisfaction that the answer belonged to an engineer, too.

``And anyway, I have telephoned with a local who confirmed to me that it is safe. The only trouble is that it is quite inaccessible---expect a fifteen mile round-trip in a rocky terrain from where the locals can take us by car. Is that a problem for you?''

``No, I think I can manage that, but aren't you getting ahead of yourself?''

He looked at her intently for a while, with a kind of a mischievous smile she had never seen from him before---the knowing grin thrown by a scoundrel to his accomplice, hinting at a secret they both understood, but would not name. ``I believe, Elke, that you want to make that photograph, just as much as I want to see it made.'' He paused briefly, his expression turning blank, and went on placidly, ``But you are correct, my lady, I presume too much. Are you interested?''

She took a deep breath and looked him in the eyes as if looking at a devil asking her to sell her soul. ``I am, Luca.''

\sectionline

Coils of rising steam dotted the furrowed, geothermal landscape of Krafta. Elke thought to herself that if the devil did exist, he'd surely love this scenery. A cold wind kept blowing her hair into her eyes and she had given up on brushing it away, again and again. She focused her eyes on the ground, watching her footing carefully as she followed Luca's swift pace through the rugged countryside. He would slow down and stop periodically, to align his map with a compass.

She had agreed to the photoshoot, but now found herself feeling increasingly nervous as their journey progressed, knowing what awaited her at the end. She was astonished to realize that the silence had suddenly become unbearable to her, and that she wanted him to say something, anything. They had always been so at ease with each other, and silence never bothered her before. He went on without a word, however. As his sole attention toward her, he would, every now and then, turn around to see if she was still following---his face empty, disinterested.

The silence left her alone with her thoughts, and his face revealed to her that it wasn't the photograph at all that she was nervous about. It was him. She had fought to admit it to herself, but could deny it no longer---she wanted him to see her. And more than that still, she wanted to see him want to see her. But there was not a hint of that desire in his face. Elke went on, crestfallen, her steps growing heavier with disappointment.

When she looked up again, she saw him stop at the top of the ridge and put his map away. He was waiting for her to catch up, smiling for the first time that day. She felt the heaviness leave her steps and walked faster to join him at the top of the hill.

``The spring,'' he said and pointed down to the other side of the ridge. In an uncharacteristic gesture of closeness, he raised his hand and brushed the hair away from her face. His eyes lingered on her a shade longer than required. It was a look of peculiar, soft acceptance, but she could not tell what it was that he had accepted. For a moment they stood still, looking at each other as if they were both yearning for reassurance.

``Come,'' he turned away abruptly and started the descent toward the spring.

Elke followed him, puzzled, yet content. She hadn't seen from him the look she had longed for, but she had pierced through his apparent indifference and was certain that she saw it for what it was---nothing but a cloak carefully worn over something profound being stifled underneath.

\sectionline

The hot spring looked like a puddle of steaming milk in an orange-tinged crater. Elke was standing at its edge, dressed in a bluish bathing suit. Remembering Luca's speech about volcanic springs, she dipped one foot cautiously in the water. It was pleasantly warm. She waded in waist-deep.

It took them some time to find a suitable spot. Elke sat herself down at the very edge of the cloudy water and rested her body against the rock that was rising with a gentle slope behind her. Then, lifting her head, she looked up toward Luca.

He was seated on a ledge in an outcrop, watching her impassionately, his hands resting in his lap.

She reached to her back, removed the top of her suit, and tossed it imperiously aside, away from the scene. With her head held high, she looked back at him.

``You look lovely,'' encouraged Luca in an earnest voice.

Without answering or showing any reaction to his comment, Elke assumed a longing expression and lay back on the rock, dropping her head lifelessly to her side.

``And you've got the knack of posing now. Flawless,'' he added jovially as he rose to his feet.

Elke lay in the spring with a feeling of strange weightlessness. She felt the water vapor rise around her and a much colder air brush over the top of her body. Luca's steps and the clicks of a camera shutter were reaching her ears, then she heard a voice telling her that she can rise from the spring now.

She rose slowly, as in a stupor, but was jerked out of her thoughts in the instant when she felt a firm touch on her shoulders. Luca wrapped a towel around her, and not saying a word, he turned back and walked to the side, letting her dress.

\sectionline

Luca was seated at a table of an Italian restaurant, laid-back, tasting his wine. The dinner had been first-class and the wine's quality more than matched it.

Sitting opposite to him, Elke was studying a printout of the mermaid photograph. ``Luca, I am glad that you persuaded me to go to Iceland with you. The scene is marvelous.''

She leaned forward, put her name on the release form, and pushed it across the table toward him.

``I am pleased that you feel that way,'' he replied, laying a stack of Euro bills in front of her.

She paused to look at the bills, pondering a private question of hers. ``In which section do I list this money on my tax return?''

He narrowed his eyes, and looked at her as if she had asked about the best caliber of a handgun to commit suicide with. She counted the seconds of silence before he spoke again, his voice toneless and somber. ``Elke, do not report this income.''

``That is completely out of the question.''

``If you are concerned about the consequences, that is quite needless. You are being paid in cash. There is no way any government thug of a tax collector can ever hope to find out what you are doing. There are thousands of laws in Germany, maybe even tens of thousands. How idiotic is that? There is no way the bureaucrats can enforce them all. They have to rely on carefully conditioning their victims to self-police themselves. It's the law, all good citizens follow laws, I am a good German citizen, I'd better do as well. That is how you think, isn't it? And it's the law to pay your taxes. Do not let them get away so easily, Elke. Do not be your own slave-driver.''

``Luca, I've come to agree with you in a way that what the government does is unjust. But what you suggest is not a solution. If you do not pay your taxes, that will just make them increase the load for the other people.''

``That is unfortunate, but they have the same choices open to them as we do. If enough people refused to pay the taxes, the charade would end.''

``For Heavens, is that what you are hoping for?''

``No. It is a pleasant fantasy for me, but I do not expect it to ever happen. My only concern is to live in such a way so that I can bear to look at myself in the mirror each morning. All governments, and the welfare state in particular, are a terrible evil. It is a moral duty of every man to evade paying taxes to the utmost of his ability to do so, lest he become an accomplice in that evil.''

``I think I'll need another drink. This is extravagant, Luca, I cannot accept it. I am not an accomplice in anything. They hold the guns.''

``Yet you have built those guns, and paid for the bullets besides that.'' He looked at her with his expression changing into a teasing smile. ``What if I do not want you to pay the taxes on the photographs? Does that mean nothing to you?''

``You have no say in that,'' she threw back defiantly. ``I am not a cheater.''

``What you say is a contradiction in terms, one cannot cheat a government.''

``How do you figure?''

``A man who willfully violates the rights of other men has all of his rights forfeit, Elke. The government has initiated the use of the force to extort from us, violating our property rights. From that day on, we are absolved of any requirement to be truthful or honest in our dealings with the government. If a thug waylays you and takes your purse, you have no moral duty to truthfully answer his question whether you have any jewelry on yourself or not. Nor have you any duty to answer such a question to the tax collector. You have been conditioned to think so by thousands of years of brainwashing---from the tithing of the clergy through the devilish speeches of Goebbels to the welfare-peddlers of today, selling self-sacrifice as a civic virtue. And people have been buying. Excuse me, sir, you've raped my wife. Here, take my daughter as well. And please accept my apology that I only have one daughter to give you. Disgusting.''

He paused to pick up the photograph, and tossed it in front of her. ``Look at that photograph, Elke. Look at that fragile beauty and remember the courage it took you to remove the bathing suit. Was that easy? Do you understand what you paid in that pool? And now some government clown comes along with, `Nice picture, lady, we're taking a slice.' A slice he has not had to move a finger to earn, then to be graciously redistributed to any bum who'd only have to prove that he needs it. Is that what you strove for?''

``No,'' she moaned in quiet dejection.

``Do you see what you have already given them? Be it in that pool or in your laboratory, the gift is the same. It is the greatest gift a human being can present to others---the sight of greatness she has given birth to. A gift they could not hope to reproduce themselves if their lives depended on it. As, in fact, they do. You will not get as much as a thank you in return. They will do their best to not give you anything.''

His lips tightened into a bitter smile. ``And it is not just the government, by any means. In more civilized times, you would be robbed by men with no pretensions about what they were doing. These days, witness the rampant intellectual property theft---every worthless scumbag can consume a great work unpaid for, proclaiming haughtily his moral right for the loot and your duty to produce more of it. A true son born of the welfare state---practicing in retail what his begetter has mastered in wholesale. They will demand that from this day on, by virtue of your ability, it will be your duty to undress for them, and their right to consume that sight by virtue of their need for it. Is that what you strove for?''

``No,'' she cried out in a tormented voice, his sentences hitting her like lashings.

Luca waited for a moment, then went on slowly, in a gentler voice. The change reverberated in her ears like the thud of a whip falling from his hands to the floor. ``Do you see, what evil you would support? There is a wide-spread molestation going on in the world, Elke---allowed, backed and carried out by the government---the rape not of your body, but of your mind. Without asking for your consent or permission, they will slice away at the products of the most brilliant of your triumphs, hopes and dreams. Yours to be no more. Why? Because they are yours, they are good, they could not have made them themselves, and they need them. Do you understand that, Elke? They will rob you of your achievements as a reward for standing between them and their starvation on some empty, barren wasteland.''

He thought he heard a deafening cry of tortured refusal, but he knew that he had merely imagined it---as something he'd naturally expect to accompany the change in her expression. She did not say a word. She sat completely still and silent, staring at him, her eyes lighting up with hostility. Only the lines of her neck muscles pulled tight revealed an intense struggle to refuse to hear, and to believe, a statement of pure, naked evil.

``The rapists are safe though. They know people like us would never, for their love of life---and for their love of that love---stop working, and in doing so supporting them.''

He paused and hesitated. He had already told her more than he wanted to, should he speak still further? He did not know which particular part of what he had said reached her, but he felt grief now that it had to reach her as strongly as it did. He saw the delicate figure sitting in front of him, her body apparently stopped from collapsing onto the table only by the waning power of a defiant will.

He ached to continue and tell her everything, but could not bring himself to do it, deterred by one more glance at the line of her slim shoulders, straining to keep taut. No, it would be like stepping on a frail plant struggling to rise against a cruel weight of a rock pressing her down. He could not indulge in his own desire to make her understand---and to make himself understood---at such a price.

The clear, controlled voice rising above the murmur of his thoughts astounded him. ``And they are right,'' she said. The sentence carried a challenge, so natural to her, but with none of her usual playfulness. It had a weighty tone to it, like a court's final verdict would.

Luca met her eyes and held her for a moment. His expression showed pure, genuine respect. She had decided his dilemma.

``But they think that we will put up with any injustice. That for every morsel of the fruit of our labor which they take away from us, we will find a way to work a little harder, or a little longer to make up for it. That we will take the veiled murder they've prepared for us as a fact of life, and find a way to make it work for them. But they are wrong about that!

``All the welfare peddler cannibals have been living on borrowed time. Fed from a loan we have generously, and imprudently, bestowed on them. I think the time has come to call that loan, Elke. Sadly, the situation has become far too cozy for our debtors, and as you can gather from the guns trained at our heads, we are no longer in the position to opt out. Indeed, they have gone to great lengths to create a vague feeling in us that there is no way out and that the situation has to be accepted. But that is nothing more than a carefully crafted illusion. We can always arrange our lives so as to remove as much support for them as possible. We can choose whether we give them aircraft fuselages and electrical devices \ldots or mere photographs. And we can choose whether we let them tax those photographs.

``They think that we will do business with the muzzle of a gun. Well, I won't. All that lofty talk of theirs about the best interests of society? It is nothing but haggling over which way the gun should point. A society of traders built on consent and free exchange of values need not to occupy itself with considerations of best interests; a society of blackguards built on exploitation and sacrifice does. I do not need such a society for anything and will live quite happily in obscurity without them or their guns. Let us see how well they will live without my aircraft.''

For a brief moment, all bitterness left his features and his face radiated what looked almost like sympathy. ``You know, Elke, there is something phony about them. And sometimes, I nearly pity them, for I do not think that they consciously understand themselves what it is they are doing. They think that we will pretend, for their benefit, not to notice the gun, and deal with them as with men worthy of respect. It is as if they begged us to make them believe that what they take from us is not looted but earned, and that they, in fact, are deserving of respect. Do they truly believe that we will help them stage that kind of a fraud on the world?''

His eyes started to lose their radiance and there was a moment of heavy silence, then Elke saw the look of implacable purpose return to his face. ``I won't!''

The memory of a wistful smile and a pained, bitter voice of a man sitting next to her, gazing longingly from a train window, shot through her mind. She looked at him with eyes wide with horror, feeling rising within her chest a sense of outrage at an injustice which she was still desperately trying to deny.

Luca stared right back into her eyes with total serenity. ``Keep the money, Elke. It is yours. Every last cent of it. You have earned it. They have not.''

She lowered her eyes toward the stack of bills in front of her and kept looking at them for a long time. Luca relaxed back into his chair and watched her calmly, sipping his wine from time to time.

When she spoke at last, it was with a lifeless voice, as if something inside of her that she had trusted had just been crushed---and its dead carcass revealed that it had been a fraud all along. ``Luca, if we do not pay our taxes, that leaves us as leeches, too. Every time we use some public service.''

``No. Not every time. There is a small number of services which we cannot opt out of, national defense for example, and it is unfortunate that we cannot selectively pay for those. But what we pay in indirect taxes, to the extent that we are consumers in society, far exceeds our part of the payment. As to the rest? Most, I'll never use---why should I foot the bill?

``And even in the rare cases where I do use them \ldots I didn't ask the government to provide those services. They did anyway, wiping out all private competition, thus forcing me to deal with them. I do not feel obliged to pay for those services any more than I would pay a mobster who's used his goons to shutdown all competitors, leaving his diner the only one in operation. If the government nationalizes a railroad, which I had been perfectly happy to use, or provides subsidized train fare, is that really a claim on me to obediently stand and watch while I am being extorted, to the extent that I am a producer in society, to pay for those subsidies?

``I am free never to use the train again in my life, of course, and if I do, you are right to think of me as of a leech. I am a leech openly, and will not hide behind a shroud of moral high ground. This is a war. And it was the government who started it. Apparently, they think it is justifiable to deal with other men by means of a force. Very well, I accept that tenet. Their force against mine. Let them catch me and punish me, if they can. Then I will go to jail. But until they do, the leeches will have to taste some of their own medicine---with me as a free-rider on their subsidized trains. Let them learn on their own skin where the use of force inevitably leads.

``But this is all rather theoretical, in the vast majority of sectors, in health care for example, private providers still exist. They are a mere shadow of what they would be without the government's interference, but they do exist. The rational and moral thing to do is not to use public services and pay for them in taxes---it is to evade paying taxes and purchase your services on the free market, or on what is left of it nowadays.''

She listened attentively, and remained silent for a while again, struggling against some lingering internal doubt. ``But suppose something happened to you that was completely out of your control---maybe a serious disease---and you cannot afford the private health care costs. Paying your taxes and having a public hospital treat you would save your life.''

``At what cost, Elke? Remember, the government cannot give you anything at all, that it has not stolen from someone else before. Since you pre-suppose that I cannot afford the treatment, that means someone else is paying for it in the public hospital. Taking that treatment would put me on the same moral ground as a man who bought a Jewish villa from the government in the 1930s, for ten Pfennigs on a Mark, never asking himself from whom and by what means it was extorted. I would not want to be cured of my disease at that price.''

``You value your principles above your life. But suppose still further that it is not your own life that is at stake, suppose your wife is dying and needs help. Will you not regret evading paying your taxes then.''

``No. And I do not believe that you would want me to feel that regret, Elke.''

Me? Is this who he thinks of when imagining his wife?

Missing her sudden look of surprised affection, he continued. ``Ideally, you'd have a private health insurance coverage, but the government's meddling has made those hard to obtain. We could, however, always raise the money for a private treatment ourselves. We could borrow or beg others to help us.''

``You would beg?'' Elke blurted out in disbelief. It wasn't the inconceivability, but the offensiveness, of the idea that had shocked her so.

``That is right, I would. I would go to every single one of my friends, then to every one of yours---and to all of our neighbors and to random strangers as well if needed---tell them what happened to us and ask for help. Not like a bum flaunting his need and his inability to afford a cure, claiming it is therefore his neighbor's duty to obtain it for him---by virtue of him not needing it and being able to afford it. I would ask like a man who knows his exact position, of not being able to repay by anything else but gratitude. And I would not expect help for any reason other than compassion. Compassion for a couple who, for their whole life, carried their weight as honest, decent people, but are now to be crushed by a cruel fate beyond their control. The value derived from the help by the one helping us being the feeling of satisfaction at preventing that fate. The thought, `It could have been me' passing through his mind.''

She listened to him speak with such strange warmth and fondness, like none she had ever felt before in her life. She wanted to see him at her feet once. Now she heard that proud, self-possessed figure say that he'd willingly get on his knees \ldots to beg \ldots and for her sake.

No, you would not be there kneeling alone. I would share that with you, and would do the same for you. It is this world of yours that makes sense \ldots but will I ever be a part of it?

``This is how people lived for centuries, Elke. Until the second dark age descended on the world at the birth of the welfare states. People lived in communities that helped each other in times of need. Out of compassion for a decent neighbor. Now the welfare state has destroyed both compassion and decency---because generosity and charity are degrading, they should be administered as a matter of social justice at the point of a gun. No one cares what happens to his neighbor anymore. Now that he is paying his taxes, other people's fate is somebody else's problem. And no one cares to be a decent person either. Why should he when his neighbor will be taxed to bail out a rotter and a saint alike? Do you see how monstrous that is? A man who indiscriminately grants charity to vice has corrupted it and has now none left for virtue. Justice has been destroyed---to the blowing of trumpets and cheers to the tune that social justice has been served.''

He stopped talking and emptied his wine glass, as if trying to wash away a feeling of disgust in his mouth. ``But there is something still infinitely more precious that has been burnt on the altars of the welfare states---personal responsibility for one's life. It is what makes life so satisfying, when you can point your finger at something you have done and proudly shout for all to hear: It was my responsibility to do this, I did it, it is mine and it is good.''

Elke was listening as in a dream, her thoughts darting back and forth from the man seated in front of her to the memory of a lonely girl sitting late at night in her laboratory, longing for just one mind to see her, and to understand.

``Now responsibility is disappearing from our lives and in the few areas where it still exists it is diluted and with a bitter aftertaste. So many regulations have been passed that no one can even keep track of them anymore, lest people act otherwise than some enlightened bureaucrat deems best. And so many safety nets have been constructed, paid for by extortions from the responsible, so that the irresponsible can be divorced from the consequences of their actions.

``Men have been reduced to half sacrificial animals, half supplicants thirsting for the last droplet of blood to flow from the macabre altars of the others' sacrifice. Neither of them can proudly point their fingers at anything now; the former for it is no longer theirs, the latter for they were not responsible for it. And both of them are caged, not allowed the freedom of unconstrained pursuit of happiness which ought to be theirs by birthright.''

The way he was playing with the stem of his now empty wineglass made her expect him to fling it across the room at any minute. But she saw the ruthless self-control of his arm put the glass smoothly down to stand on the table. They held each other's eyes for a moment, then Elke leaned forward, grabbed the Euro bills from the table and stuffed them into her purse with an unmistakable gesture of a woman handling her private property.

\sectionline

It was a cloudy evening. The murky sky gave an air of the obscure, the indistinct and the slack to everything beneath it. Elke turned away from the window, the sight had made her shiver uncomfortably. She had always used the memory of the clarity, precision and discipline imparted by her laboratory as a shield against such feelings. But she dared not think of that now, her team's daily operations have not recovered fully yet from the tumult of the inspection.

She sat down into an armchair and turned on a music recording that she had picked up on her way from work. The music did not interest her, however. Nothing interested her right then.

She got up and paced her apartment. She had not met Luca for three days and her uncertainty about the nature of their relationship was returning, as it did with various intensities on many occasions during the last few weeks. She had fought it as best as she could, but could fight no longer. She would find out tonight either way. She put on her nicest blouse and a tight skirt, as he had told her once that he'd like to see her, and called for a cab to take her to Neuhausen.

As she walked toward Luca's tenement house, she stopped to observe the rooftop unit. It was dark. Doubts filling her mind ever more forcibly, she took the lift. It's no use, he's not going to be at home, thought she while passing through the hallway leading to his apartment, slowing down with every step. She recognized her anxiety then, and stopped angrily, shaking the feeling away. She took a breath, raised her head high and walked swiftly on. She reached his door, pressed the doorbell and waited with suspense.

She heard a noise through the door, and another---he was not away. Then the door opened and a visibly surprised Luca appeared. ``Good evening.''

``Good evening, Luca. Are you busy? I wanted to see you.''

``No, I am not busy. Come in,'' he said, stepping aside to let her in.

The apartment was dark, the only light illuminating it came from a computer screen. A strange music filled the room. It only dawned on Elke now that she had not thought about what precisely to tell him. ``What music is that?''

Luca joined her in the room and switched on the light. ``It's Italian. A modern adaptation of traditional songs from Calabria, in the very south.''

They remained silent. She saw him studying her dress and swayed her body, emphasizing the fact that her selection was not random. Then, she daringly met his eyes. He held his face tight, emotionless, but his eyes betrayed a conflict raging within. Elke saw it and understood its nature. She was looking at a man fighting his desire to enter a cage full of tigers. She felt a sense of impudent pleasure at the sight of his torture and throwing him an impish smile, she turned her back on him, walked toward his desk and leaned against it.

Moments of anticipation stretched on for Elke, with only the subdued tones of the Italian song marking the passage of time. At last, she heard his steps behind her and knew that he had lost his fight.

With sudden ferocity, he pushed his body against her. She felt the warmth of his breath on her neck, his lips hungrily meeting hers and his hands reaching under the fabric of her blouse. The pressure of his hands on her skin was harsh, tormented---rather than pleasure seeking to convey animosity and exact punishment. Punishment on her, for so obviously desiring what he had damned himself for desiring, and what he wanted her to be above desiring. And punishment on himself, for not being able to resist giving it to her.

He twisted her body to face him and pressed his mouth on hers in a drawn-out, tortured kiss. Her eyes laughed at him with defiance. That poise and monk-like discipline of his \ldots brought down to this \ldots and by me. A low groan escaped his lips. She had wanted to hear that one groan ever since their Greek dinner meeting. She grabbed his head and pushed it down to touch her mouth again, feeling the pride of ownership of his body at its every touch of hers.

Abruptly then, he tore his mouth away from her, and pulled his hands back. In the next instant, asserting in turn his ownership of her, Elke felt him press her body down to bend over the desk with such force that she found it hard to breathe. He did it with demanding brusqueness, as to tell her that he needed no permission from her for anything---she had given him that permission by coming here tonight.

She tried to push herself up on to her elbows but could not raise her body an inch against his strength. The briefest flicker of fear filled her heart---any resistance against his will was futile now. But then she gripped the edge of the desk in breathless anticipation of being turned into an object for use by that will. She felt him lift her skirt and then only the rhythmic violence of his body beating against hers, accompanied by her half-broken moans between hurried gasps for air.

Suddenly, the rhythm stopped and Elke felt his body leaving hers. She kept lying on the table, fatigued and limp, not making a single move. There was no desire left in her to fight the fatigue. She hoped it would last forever and she'd never have to move again.

He pulled her skirt back down and slowly ran his hand over her back. His touch suggested tenderness, his voice anguish, ``Forgive me, my lady.''

She stood up, and straightening her blouse, turned around to face him, her expression calm. ``Did you want it as badly as I did?''

He stepped closer, and ran his hand through her hair in a caress. ``I have wanted you since the first moment I sat in that Marburg diner and heard your blood-red lips pronounce a creed more sacred than any that has ever been uttered on this Godless Earth. And now I have hurt the brightest flame that I have ever seen in the darkness of my life. You, whom my strength ought to have been protecting from hurt.''

``You didn't hurt me, Luca.''

``Didn't I? That makes my crime worse still, for I will.'' The sadness in his voice was changing into loathing. Elke could not decide whether it was directed toward her or toward himself.

``Every day of my life, I have paid for my way, claiming that all men, inasmuch as they want to be men, must. Now I have betrayed that and turned my back on my whole life. I have wanted you, and I have taken you. Not like a trader buying goods he has earned, but like a looting bum gorging on that which is not his without even inquiring whether he can afford to pay. I do not know what you expect from this relationship but I suspect that I will be nothing but a penniless beggar unable to give it to you. Note this, I will never live with you, I will never marry you, I will never compromise for you, and above all else, I will never work for the kind of world endorsed by you. May you damn me for this!''

Elke listened very attentively. Her burning anticipation, completely consumed in their act of intimacy, was being replaced by a sense of easy calm and contentment. She watched the tall figure standing in front of her, taut and openly unprotected, as if expecting her to slap his face and indeed inviting her to do so. But the glimmer of guilt in his face drowned in the glare of pride---the pride of knowledge that even this guilt and whatever punishment she may choose for him were his to endure; and he would endure them, as he had endured the pleasure of using her body.

She tried to examine, as he had asked, what she expected from the relationship, but could not answer the question herself. For another thought was demanding her attention---the realization of her true purpose in seeking his company today. He was wrong that this hadn't been a trade for both of them.

Her eyes kept glancing over his body, as if driven by a will of their own. It was a body she now owned through hers, and it gave her intolerable pleasure just to watch it: the austere patience with which he waited for her reaction, the unflinching eyes that would not evade hers \ldots she had to laugh inwardly at the absurdity of his self-damnation---this was no way for a beggar to stand.

In a moment of lucidity, she felt pieces of a riddle clicking together in her mind, answering a question that had been puzzling her since Nürnberg: So there is a way to pay \emph{for all things}, and this is how you've paid for me.

``Luca, you are a trader. You had paid for that which you just took many weeks ago \ldots no, years ago \ldots by becoming the man who drew me here tonight. That a man like you would admire my mind and hence want my body, that is your payment to me. And I can imagine none greater. Note this, I want you and I do not want anything from this relationship which you will not give me freely. If that particular kind of exchange makes me the cheapest trader in the world, then may you damn me for this!''

She had barely finished the sentence when she felt his hands wrap around her and his mouth find hers.

\chapter{The Ice Queen}

\epigraph{Any intelligent woman who reads the marriage contract, and then goes into it, deserves all the consequences.}{Isadora Duncan}

Their meetings had become more regular now. He would wait for her to finish at work and take her for dinner, to spend the evening afterwards together. She wondered if she only imagined he looked content, his outbursts of bitterness having grown rare, or if he had merely learnt to hide them from her better.

The first time he came to visit her in her apartment, she greeted him with ``How do you like my warehouse?''

He looked around her living room with interest, taking special note of the stacks of musical recordings lined next to one of the walls. The air of purposefulness pleased him very much. He had to concede that there was not a trace of disorganized clutter to be found in the room. Then his eyes rested on a vacuum tube amplifier sitting on one of the shelves, obviously hand-built. He turned toward Elke with an admiring look. She nodded proudly, smiling, and switched it on. Rial's music filled the room.

His hand trembled slightly as he reached out to brush a lock of hair from her forehead. He had to admit that at this very instant, she was the stronger of the two of them. He could allow himself to surrender to that feeling this once; to use her strength for support that he had not realized before how painfully he needed. That support, however, did not radiate from the thin, slender figure standing in front of him. It radiated from the mind of quiet competence, and from the mellow singing which that competence had enabled to fill the room.

Elke was gazing straight into his eyes throughout his ruminations and had no doubt that she had guessed his thoughts. This was the moment she had been waiting for---her whole life long---of another mind seeing hers, telling her that he saw her, and that she was good. A moment of stripping her achievement naked for him to witness and admire, a moment of intimacy between them so profound that it utterly dwarfed any they had ever shared before.

Luca ran his hand down the length of her arm, allowing the hand to exert only the slightest hint of a pressure, as if afraid to disturb the preciousness of the moment by a physical touch. He held her gaze for a while and nodded his head solemnly---confirming to her that she had guessed at his thoughts correctly.

Then, he took a step back and spoke in a lighthearted tone. ``Your warehouse barely qualifies. I do not think that you are stocked well enough here, mister storeman. It is good that I've come to replenish one of your recent losses.''

He handed her a soft package and sat down into an armchair, watching her. She tore the wrap open to discover a dark-purple sweater made from the softest cashmere.

``To replace the one ruined by our chief-engineer.''

She looked at him with an appreciative look. Seeing him watch her intently, she removed her top, accenting every move for him to notice, and slipped into the sweater. He rose to his feet, embraced her and kissed her lips.

``Luca, you've never done any group model photographs?''

``No.''

``Why not?''

``I never felt the need to. You have something specific in mind?''

He had been observing Elke's growing pleasure with photography for some time. At first, he attributed it to her discovery of a desire to display herself, or even to shock the world by the way she chose to have herself displayed. But he was certain now that there was more to it---she enjoyed thinking up scenes, choosing make up and selecting props. And she was good at it.

``Yes, I was playing with the idea of a simple contrast of black and white. Of a dark haired woman clad completely in white along with a blonde in black, maybe a goth teenager---not a natural blonde, but with hair colored an artificial shining white. Both with the sternest expressions they can manage walking hand in hand. I can't think of a location though.''

Luca was listening to her description, grinning. ``You already have plenty of contrast in the scene, the background should be something undistracting, or even shot completely out of focus, with only the models sharp.''

``You like the idea then?''

``I love it.'' Then, pointedly looking at her, he continued, ``I see we already have the first model. You might, however, have a whole lot of fun looking for a blonde goth.''

\sectionline

They spent the nights of the next week visiting dark music halls and metal concerts.

``I feel so creepy checking people out like this, Luca.''

``If you feel creepy now, wait once you're asking her out on a hand-holding date.''

``Me?'' she exclaimed in horror. ``We'd lose her. You ought to do it, you can wheedle anyone.''

``I will take that as a compliment,'' he replied smiling.

``It was one.''

They finally found the perfect model in an underground music club, sitting with a group of friends.

Luca's calm, sonorous "Good evening" rose clearly above the music as he approached the group. Not waiting for an answer, he seated Elke at the table and sat down next to her. The goths stared at the stranger wearing slacks and a sportscoat with incredulity, but Luca didn't appear troubled and continued to talk quite casually.

Elke sat timidly by his side, unsure of what to think. Their out of place clothes, Luca's uncharacteristic break of courtesy in taking a seat uninvited, and the incredulous glares she now met at the table had made her feel nervously awkward. But then she glanced at Luca and in an instant the feeling disappeared, subdued by the radiant certainty of his face. It was a face devoid of even a hint of guilt, of a man who'd laugh at a mere suggestion that he had anything to apologize for, or expectations of any kind to meet.

``On the day that you understand that I do not care about your, the society's, or anyone else's estimate of my attitude, then you will have understood that attitude,'' she recalled him telling her in the past. His complete disregard of her had irked her then, making her wonder how low one has to sink to stop caring for all judgment other than his own. But looking at him now, she realized how wrong she had been---this was the height to which one rises once all judgment other than his own ceases to matter.

It was so easy and natural to be truly, openly herself around him. In his presence, it was not necessary, indeed not even possible, to think of any pretense, or to consider what someone else might think of her. To be herself and to be alive. She felt blood coursing through her veins with every beat of her heart. This was the only way to be alive. A sudden desire to pay homage to this kind of fierce aliveness overcame her then. She wanted to have him---and she wanted to be had by him---as the only suitable tribute to this moment which she was so glad to have lived. She dared not turn her face and look at him, fearing that even a single glance would be too dangerous a test of her ability to resist.

She knew that the others at the table felt as alive as she did; the glares of amazement had vanished and they eagerly talked to each other now. Elke found herself laughing cheerfully and answering questions without effort. Her confidence and self-control had returned and she kept glancing at Luca. Despite the club's dark setting, with only candles set around the table for illumination, she couldn't help but think of his face as if it were hit by a beam of violent light. No, there was nothing out of place about him. He belonged to this place, as he belonged to a luxurious restaurant, to a medieval city, to the ruins of an old factory or to the wilderness of Iceland. Could I, too, belong \ldots at his side?

She was yanked out of her thoughts by a change in the tone of the conversation. A young man with multiple piercings on his face and a spiked collar around his neck, which appeared to choke him slightly, had spoken for the first time. There was veiled hostility in his words and she found Luca switching to his lecturing tone in answer.

``If you really believe that, then I suggest that you examine what difference there is between a corporate man wearing a tie to please his boss and keep his job on one hand, and a young man wearing a spiked collar to please his friends and keep his clique on another. You will see that there is very little to gain in exchanging the tyranny of society for the tyranny of a label, and likewise what it is that your vaunted freedom truly amounts to.''

Elke's hand shot out to reach for his arm, her alarmed expression like a command: Drop it; this is no way to talk at a party.

He turned to her with a flare of defiance, ready to challenge her command. After a brief contest of wills, however, he nodded, smiled softly, and took her hand in his. There was no reluctance in that nod and the willingness with which he changed his tone back to casual friendliness was like a message telling her: But of course. You are right. And I do care about your judgment---not because it's yours, but because it's just.

``I apologize,'' he said to the man. ``This is your table and your rules. I respect that. Anyway, we haven't come here to discuss fashion trends with you and will not take more of your time. We only wanted to say a couple of words to your friend here.'' He turned to the blonde girl and introduced Elke. Then, he let her describe her photography idea. It wasn't hard to persuade the girl.

``See, I told you,'' patted him Elke on the back as they were leaving the club.

``Told what? You did all the talking, I merely introduced you.'' He slid his arm around her and drew her close, his admiration stressed by the part protective, part craving touch of his hand on her skin.

She felt the need to pay tribute to the moment returning to her body, knowing fully what kind of introduction had made her success possible. Yes, I do belong---at your side, her thought started---in your bed, it finished. We have both earned it tonight, through your value and mine.

They shot the photograph the next day in Schlosspark Nymphenburg. It sold within a week.

\sectionline

The weekend trips had grown longer, they both staying together in a hotel room overnight. Berlin, Hamburg, the industrial Ruhrland, Luca would find something fascinating about any place and Elke had come to trust him to do so even for destinations that sounded bland at first.

Lying in a hotel bed one night after a lovable boat ride on the Alster, her head resting on Luca's chest, his hand stroking her shoulder, she asked, ``You do not want to live with me, Luca?''

He looked down at her with the resigned expression of a man who is finally handed a bill he knows he'll be unable to pay. It lasted for only a brief moment, however, total calm returning to his face.

``No,'' he replied tonelessly.

Elke felt a stab of pain at the unqualified answer and prodded on, ``Is it disagreeable to you \ldots the way we are right now?''

He lifted her from his chest and turned over onto his side to face her. ``It would be, if it lasted too long'', he said running his hand gently over her arm.

``I am not built that way, Elke. I do not think that most men are, but they had suppressed that in themselves somehow. Look at all the palaces and castles we've been visiting together. The wealthiest and the most powerful men in history built their houses with separate queen's and king's quarters or something equivalent. The aristocracy, no longer able to afford such an extravagance, provided at least for separate bedrooms. It were the poor townspeople being forced by necessity to a close living space who made a virtue out of that necessity.

``What would we gain from that, Elke? Do you not have your own life and passions? Do you need somebody else's permanent presence to entertain you? Do you want to worry about that someone squeezing the toothpaste the wrong way all the time? I think there is a huge number of couples sentenced to a lifetime of frustration and disappointment, who would have made amazing life partners had they only lived apart. Let us live our lives freely and independently, meeting where our interests and passions overlap, when and only when both of us desire to do so.''

There was something in his speech that resonated deeply with Elke. She thought of the man of her past accusingly pointing out her lack of hobbies. Wasn't he just condemning himself for the lack of his? It was the unspoken, but implied, insistence that it was somehow her duty to entertain him---by virtue of her not needing to be entertained---that subconsciously irritated her then, although it took years until she now heard Luca put into words the reason why it did.

``You might have a good point there, Luca.'' A playful smile was forming on her lips. ``I wouldn't want to force you to live in a warehouse.''

``I'm glad that you see it that same way I do. I wouldn't want to force you to sleep on the floor, either.'' He threw the blanket off of her and pulled her body closer.

``Do you see what a wonderful thing we have here? You are everything I can respect, honor and admire---and I hope that is also what I mean to you. Do not let it fall to some uneasy compromise of duty, loyalty and sacrifice. We are not a couple, a relationship, or a household. Such collectivist chimeras have no mind, no hopes and no desires of their own, and thinking of us that way is a sure way of replacing the wonder with burdensome obligations. No, we are two sovereign individuals who have found comfort and enjoyment in each other's company. I want to spend time with you on every mutually satisfying activity that we can think of---and not one minute longer. The rest of your life is yours to spend whichever way you desire. It is \emph{your} life, Elke. Yours to be lived, not sacrificed to please someone else---not the society, not your relatives, and definitely not me.''

No, thought Elke, it is not possible for him to hurt me. Not as long as he talks and makes sense like this. ``And this from a man who thought himself penniless,'' whispered she and kissed him.

She felt his tongue hungrily seeking hers, his hands circling her body, to at last grab her hair in a rising violence she had been waiting for with anticipation.

\sectionline

Luca entered the restaurant he was to meet with Elke in. She was already seated at the table he had booked. He paused, enjoying the sight of an elegant woman waiting for him. She noticed his arrival then and rose from her chair. He smiled in acknowledgment and approached, kissing her as a way of greeting.

She smiled at him lightheartedly pointing with her eyes to the already filled glasses on the table. ``I took the liberty of ordering a drink for you.''

He took a sip. ``An excellent choice.''

The evening passed pleasantly, he inquired about her day at work and Elke embarked upon an excited explanation of a particularly challenging problem which her team was facing. ``Oh, but let's leave it at that. Don't get me started on signal processing. I'd never stop talking then.''

``I'd listen to such a keen talk any time of day, on any topic.''

``What about your day, Luca?'' she replied, beaming with joy at his answer.

``I was contacted today by the magazine we did The Mermaid for. They'd like another nude photograph.''

``Which theme?''

``They don't care, as long as it's nude.''

``How trite.''

``It is. But maybe we can cook something up and oblige them. What do you think?''

``Don't pretend this is a question, Luca. You know full well that I won't be able to resist. And I think you've cooked something up already,'' she said, laughing at him.

``I was thinking that we can go back to Iceland and make a photograph of the ice queen, with a bit of an unusual feature. There is a glacial lagoon in the south of the island, called Jökulsárlón. It's a pretty busy tourist attraction, but we can take a boat to the middle of the lagoon, where there won't be any audience to distract you. I have this idea of a woman lying gracefully, completely naked, on one of the ice slabs---all her features, hair, eyebrows, eyelashes, lips and fingernails, as black as a visagiste can make them.''

``I think I preferred a hot spring to a slab of ice,'' quipped Elke.

``Is that a yes?''

``I do not know, the scene does not sound complete.''

``That is true, there will be one more element. The only color in the scene that is not a shade of white or black---a large ruby resting between her breasts. Shining as if catching the last rays of a dying light in the middle of the desolation of jagged ice, stretching in all directions as far as the eye can see.''

Noticing the interest in her eyes, he stood up and walked behind her chair. Then he retrieved a shining red gemstone from his suit pocket and clasped it around her neck.

She gasped with astonishment and alarm. ``Luca, where did you get this?''

``I've rented it. For the remainder of the week.''

``That was a foolish thing to do without telling me first. You took a huge gamble.''

``I took a gamble, yes. But I do not think that it was a huge one.''

She fixed him with a stern look, but seeing the expression of gaiety and appreciation on his face, she had to smile back at him, admitting that he was right. Then, with a sigh and a reprimanding shake of her head, she picked the ruby up in her palm. ``It is beautiful.''

He sat still, waiting for her to continue. ``All right, Luca. Let's do that photograph.'' Frowning suddenly, however, as if remembering something disturbing, she added, ``You're not going to claim that one can rent a gem like this for the couple thousand Euro the magazine will pay us, are you?''

``No. This photograph, I'll be making at a loss.''

``What? Why?''

``Don't worry, you will get your usual payment. I am getting paid in different ways,'' he looked at her intently. ``By my personal pleasure at seeing the gem on your chest now, and by the one I expect to feel in the lagoon.''

\sectionline

Luca bought a pair of flight tickets to Reykjavik and came to pick Elke up from her apartment on Friday evening. He opened the door with his own key, which she had given him some time before, and entered the hallway. There was no answer to his greetings. He went to the living room, removing his sportscoat along the way. It was quiet and semi-dark there, with only the twilight of the setting sun filtering in through the windows.

He noticed Elke the first moment he stepped into the room---slumping despondently in her armchair. She raised her head, looking at him with large, sad eyes, and gave a quiet greeting, ``Good evening, Luca.''

He dropped his coat, ran toward her and sat down on the floor at her feet, taking her hands in his. ``What's the matter, Elke?''

She squeezed his hand and slid her body to the side, making room for him to sit next to her. He obeyed her open plea for support and sat down on the armchair, still clutching her hands.

``They killed us, Luca. Our work of two years \ldots it is just gone.''

He remembered the lively voice of just a couple of days ago, telling him about the project she led, and shuddered. He lifted his left hand up to rest on her shoulder, silently encouraging her to continue.

``You know the sector tax on the chemical industry that has been discussed for ages now? It has been passed. Our client immediately suspended the project he has hired us for---until they can reevaluate their changed financial situation. Suspended? In this kind of work, there is no such thing. It is ended. Dead at childbirth. And we were so close.''

The last of her words came out as little more than a whisper. She looked up at him but stopped aghast at the expression in his face. Had there been an emotion in his eyes, she would have to think he was looking at her with what was pure hatred. But there was no emotion in those eyes. It was a face of a dead man, sleeping behind some impenetrable wall making him dead to the world, and dead to her. He let go of his embrace and stood up from the armchair.

She watched him rise away from her with agonized perplexity. Was this the same man who once professed to have his strength protect her from hurt? Couldn't he see that she was hurting?

``Luca, do you understand what I am saying? They've sunk millions into the project and now won't be allowed to reap the benefits thereof. We've invested years of our lives only to see them all be in vain. What have we done to deserve this?''

She heard a steely response of a complete stranger, ``I do understand. It is you who does not.''

``Luca, why do you make things so hard for me today? Can't you see I need you tonight?''

``I will make things as hard for you as they need to be. For I refuse to nurse self-inflicted wounds.''

``What are you talking about? I am really in no mood for these games right now.''

``That is exactly the reason why you should play them. What do you think it is that you have done to deserve this punishment?''

She felt rage heating up in her at his enjoyment of her pain, and wanting to inflict pain on him in turn, she straightened up and angrily tossed back, ``We have done nothing. We are being punished because we are---'' she stopped before she could finish, terrified by her own thoughts.

The first signs of an emotion lit his eyes. ``Finish it!''

It was a command in a voice she dared not disobey. ``We are being punished because we are the best.''

``Yes. Best able to bear a new batch of leeches. So that your lifeblood may be used to feed the worst. So that you, who have the strength, the resolve and the ability, may serve those who haven't.''

He paused and looked into the distance, the distance of space but also of time. ``You told me once that if this is the price for your life in society, then you will pay it. Do you still believe that? Will you bring out that which is best in you to see it chopped and thrown to \ldots to whom, Elke? Is that the ultimate goal of all your effort---the one collector of the return on your life?''

She stared at him in silence. Her despondency was gone, replaced by a vague feeling of revulsion.

``Had you been an incompetent dilettante and the chemical industry a shameful failure one step away from bankruptcy, do you think you would be getting a tax hike now \ldots or a subsidy? Do you expect to be rewarded for your mistakes, Elke?''

``When did I ever give you reason for such an impression of me?'' Her wounded voice struggled for a semblance of defiance.

``Never. You have, however, made a terrible mistake, Elke. The mistake of inconsistency. Why don't you hold others to the same standards to which you hold yourself?''

``I do not hold others to anything; my only concern is with my own life. What standards or principles are adopted by others is not my responsibility.''

``Really?'' he scoffed at her. ``Then do not cry now if you cannot reconcile your standards with those of a world that equates virtue with suffering and serfdom. Yours is a standard of life. I'll leave it to you to name what it is that theirs aims to achieve. But should you ever accept or sanction any other standard than that of life, then there is only one kind of future ahead of you, and you know what it looks like.''

She gasped. I can carry hundreds of leeches on my back without even noticing. Her own memory of that statement seemed to laugh at her with derision and scorn.

``Did you expect to be punished by the world, for being good?''

``No.''

``Did you expect some whining bum's need to be a higher claim to reward than your years of study and work?''

``No.''

``Did you expect to be left defenseless and voiceless in the dark, by virtue of your brilliance, while the bum is free to loudly shout demands of you---and to see those demands granted, by virtue of his incompetence?''

``No!'' she shouted at him angrily.

``Did you expect---''

``Stop it,'' she interrupted, quietly but with soft severity.

She had tried to make her voice sound as a command, yet the words came out as little more than a plea. He did stop, however, and watched at her silently.

He was studying her with a look of a predator eyeing his cornered prey, and she fidgeted uncomfortably. It wasn't his look that unsettled her, however. It was the realization that her anger had vanished, and that what was slowly taking over its place within her mind  was fear. A diffused kind of apprehensiveness without any obvious reason or source. What have I done?

``Who do you think made it all possible, Elke?''

``I do not know, I am not interested in politics,'' she said, relieved by his softer tone, thinking that the danger had passed and that the conversation was taking a safer course now. ``It came from the Federal Ministry of Economic Cooperation. They are headed by Kurt something, or is it Klaus?''

``That is not what I mean. This Kurt guy, or whatever his name is, is just a puppet running amuck without a puppeteer, driven by invisible, disembodied forces beyond his control or understanding. He is not the cause of anything, he is merely an effect.''

``What are you talking about, Luca? What is the cause then, who made it possible?'' she blurted in sudden anxiety; the conversation had not become safer after all. Her questions sounded unconvincing to her own ears. The fear filling her mind told her without a doubt that she already knew the answer. Have I not condoned of it?

``Who sanctioned the racket, Elke? Who accepted it? Who kept it alive and going? Who fed the racketeers? Was it the leeches? Who built the chemical industry for them? Who built it so magnificent and so very worth looting?''

He saw a convulsion run through her body and her head drop down to her chest as she replied, her voice breaking down. ``I did.''

``The ancient Aztecs believed blood sacrifices were required to cause the sun to be raised by the sun-god each morning. The welfare peddlers of today are thoroughly persuaded of something similar. They huff and puff importantly but achieve very little for the grisly payment demanded of you. Yet for their convictions, they will not hesitate to slice your throat on the altar of the common good---that is, once you've built one for them. You lay yourself down on the altar so temptingly; would you resent them for taking that which you, in your innocence, freely offer? Why should they bother to study, to work, to think, to produce? They own \emph{you}---everything they need will be for you to create and for them to simply seize.''

He sighed and turned away from her, staring into the distance at the blackness of a city awakening to life with countless dots of electric light. The city's skyline was lighting up like a silent outcry against the engulfing darkness. His face hardened at the sight.

``For a couple of brief centuries, reason has won in the West. Men knew what the true cause of the sun's morning rise was and rejected its call for a bloody sacrifice. But we are going back. Back to the age of the savage shouting with conviction that some will have to sacrificed so that the many may prosper---with the final argument behind that claim the obsidian edge of a knife held in his hand. This is the future you are toiling for, Elke. If this is truly the price of society's survival, why should we wish for it to survive in the first place? You only have one life---of what value is society, or anything for that matter, if in order to preserve it, you'd have to give up your life or the happiness it can bring you?''

She was listening with her head still resting on her chest, staring into the ground. Have I been an accomplice all along?

Luca paused for a short contemplative moment, then whirled back to her. ``You say you need me tonight, and it saddens me to see your pain. But I cannot help you. I cannot divorce you from the consequences of your choices---and I wouldn't even if I could.''

He stopped talking and she slowly raised her head to look at him. She recognized the stress he had placed on his last sentence, and grasped what treason he would be committing by comforting her tonight. It did not lessen the pain he was causing her, but it removed from it the ugliness of the causeless and the inexplicable. She nodded her head as a sole indication that she had understood.

There was a slight movement of his lips, not enough to even hint at a smile, but enough to appear to her like a salute to the strength which she had to find in herself for that one nod.

If I had accepted this \ldots if I had accepted that it is for virtue to be punished and for vice to be rewarded, what other kind of world did I expect to find myself living in now?

She suddenly felt vaguely dirty---covered with an undefined sort of an evil, sticky goo. She sank back into the armchair with the heaviness of longing to take a bath, together with the helplessness of recognizing that there was no shower in the world able to help her wash the goo away.

In front of her, behind the silhouette of a man watching her with an air of purposeful patience, the last rays of the setting sun were gilding the city skyline, then to finally disappear below the horizon. Much more than daylight was dying for Elke in that sight.

``Look carefully at the knife plunging into your breasts right now, Elke. You have forged the blade yourself. And it is your own hand that grasps the hilt. You haven't been overpowered. How could you be? By impotence, lethargy and need? No, you have willingly subjugated and enslaved yourself---and the chains around you exist only because you've allowed them to hold you. Ask yourself someday, what duty do you have to bear them.''

He finished sadly, and with a notably gentler voice, then turned his back on her and walked toward the door leading to the hallway to turn on the light. It had become almost completely dark while they talked. He picked up his coat and looked around the room. There was no sign of her travel bag.

``Are you in a condition to make the trip to Iceland?''

She replied very quietly and after a short pause, but there was enough resolution in her answer to make him certain that she was, ``I am.''

``Then come on, it's getting late. I'll help you pack your things.''

\sectionline

Elke was able to completely shake off the thoughts of the painful conversation only on the next day, as she sat down onto a visagiste's chair.

The visagiste protested at the choice of her lipstick. ``She will really look much better for your photography with a contrasting red on her lips.''

``No, we have a different red item in the scene, which needs to stand out. Make her lips black, please.''

She was surprised how easy it was to forget her laboratory, and the pain it now evoked, when presented with an opportunity to act. The bureaucrats had not consulted her, had not even acknowledged her existence, had left her with no means of defense or action. But now she was unhindered and free to act once more.

She looked at her black features in the mirror and laughed back at the imposingly frightening face. This particular activity was something that she was quite certain she'd be able to enjoy.

\sectionline

Her certainty disappeared as they approached the lagoon. She eyed the dozens of tourists on the shore with some nervousness, and was glad to see the amphibious truck, which Luca had hired, already waiting for them. They got in and the truck plunged into the water, propellers spinning up to move them farther into the lagoon, toward the densest iceberg fields.

It occurred to Elke only now that there was to be an audience after all, the boat driver. He was a short, sinewy man with an unkempt beard and what appeared to be a permanent scowl on his face. She was trying to fight her uneasiness by telling herself that thousands of people would see the photograph anyway. It was not reasonable to fret about a single boat driver. She was not very successful, however, and could only despondently wish he wouldn't eye her so openly.

Luca sensed her disconcertion and draping his arms around her waist, led her to the railing to enjoy the view of the majestic lagoon.

``Is it not wonderful, Elke? There was no lake here as late as sixty years ago. Now the glacier is retreating inland leaving this gorge behind,'' he was pointing for her to look in the direction of the ice covered mountain. ``One day, there will be a fjord cutting that mountain and the glacier gone. We may even live long enough to see it.''

Elke leaned her head back to rest against his, grateful. He succeeded where she had failed.

But then, involuntarily, her thoughts wandered back to her laboratory and to the pain of being robbed of something equally wonderful. ``What horrors we may yet live long enough to see, Luca?''

She was shocked that what she meant to be no more than a silent musing, she had pronounced aloud. There was a determined, violent clarity in his look, as he gazed far into the distance, toward the sun. She was sure that he immediately understood what made her ask that question. It was the look of the most merciless purpose she had ever seen in her life. ``Let them do their worst. Let them create a world of horror and pain for themselves. We do not have to accept it, nor do we have to live through it. This moment, Elke, they can never take away from us. The rest is gone \ldots irretrievably. Do not look back to it.''

Yes, this moment. And also the future that it means to us---for our spotless virtue, our brilliant consciousness, our unyielding purpose. It is all we need. We truly do not need them for anything, and never did. \emph{Need?} She found herself overwhelmed by loathing and anger toward that emissary of the corrupt moral code which she had herself accepted. I do not \emph{want} them for anything! For anything that cannot be honestly earned and bought.

She closed her eyes and spoke in a quiet, tender voice the words that had never been spoken between them before---but she knew that they had both long understood. ``I love you, Luca.''

She felt his grip on her waist tighten as his head jerked toward her, abruptly and somewhat surprised. His eyes were clouded and sad, and although he was looking at her, she was sure that it was not her who he was seeing.

He was seeing the soot-covered sleeves of a girl stretched on his futon \ldots and a man standing above, pronouncing on her the verdict of an enemy. He'd give anything in exchange for the ability to wipe that memory out from his mind.

He forced that image go and focused on the woman standing before him, her black lips quivering with anticipation and longing.

Elke saw the sadness fade away and lucidity return to his eyes, as his mind raced back to the woman chastising him for a belief in an unholy God; to the excitement and the crushed despondency of a voice first sharing, and then pleading for, the joys of her achievements; to the soothing singing emitting from a piece of equipment built with limitless exactitude by a capable mind.

``I love you, too, Elke.'' His words still echoed through her thirsting awareness as he pressed his lips to lightly touch her forehead.

The boat neared an iceberg field and Luca directed the driver toward a spot that looked promising.

``Is that safe for us to step on?'' he inquired as the boat stopped and slammed gently against a largish block of floating ice.

``That iceberg is fifteen meters thick, mister. You can drive a tank on it,'' replied the driver lodging anchors into the ice and putting a ramp down for them to descend.

Luca stepped down first, then motioned Elke to follow. They walked to a flat slab of ice. She quietly undressed, trying not to look toward the boat, nor think of its driver, and stretched herself on top of the ice block. Shivers ran through her whole body. ``This will be hard earned money, Luca,'' she said, attempting to smile.

He looked at the graceful figure of the woman he loved, lying naked in front of him, her cold shivers still replaying themselves in his mind. He learnt then what it feels like for a man to find out that he is capable of murder. For that would be exactly the fate any government hoodlum would meet at even a hint that any portion of that hard earned money was to be his.

``I know, we'll make it quick.'' He stepped next to her and arranged the ruby to rest between her breasts.

She reran the scene in her head and curved her figure to match that vision, adopting an expression she'd associate with a sleeping queen, sleeping with eyes wide open. She could not think of anything but the freezing pain, rising up through her body. She did not even hear the sound of the camera shutter.

The only thing she was able to notice was the sudden warmth as a pair of hands gripped her waist and lifted her from the ice. ``Exquisite, Elke. You did not need to act at all, you are the ice queen. Now, let's go before you catch the cold.''

\sectionline

Elke sat in front of Luca, holding a printout of the ice queen photograph in her hands, having just finished the now already familiar exchange of a release form for a stack of bills.

``You know, it sort of reminds me of that goth girl who we photographed recently. I keep thinking that she'd look better in the setting than me.''

``You do not like the way it turned out?''

``Oh no, I love it. I am really surprised how well the black lipstick looks. I might even start to use it regularly,'' she laughed out loud at the thought.

``Maybe a change of image, Elke? A bit of the lipstick and you'd fit right in that underground club.''

``Well, not really my kind of crowd, nor my kind of music.'' She paused, as if considering something uncomfortable. ``Luca, do you have anything planned for tomorrow?''

``No. You have a plan?''

``My parents would like to invite you to lunch with us tomorrow.''

``Gladly.'' Then, contemplating Elke's bearings, he added, ``You do not sound very enthusiastic. I thought that you loved your parents.''

``I do. It is just that my mother is not happy about the ice queen, not happy at all. She has seen her online.''

``What did she think about the mermaid?''

``She wasn't happy about that either, but nothing like this. She can be stubborn, Luca. Can you try to be diplomatic and not tell her any controversial ideas of yours?''

``When did I ever tell anyone any controversial ideas?'' he laughed gaily.

``Oh, this is going to be a lovely family gathering.''

\sectionline

Luca, Elke and her parents were seated at a large, round dining table. The main course served was the Bavarian Sauerbraten. The cooking had made a real impression on Luca. ``This is the best German food I have yet eaten in the country, Frau Tresler.''

``Thank you,'' she replied curtly.

Her mother was tense, but polite, and Luca seemed to get along quite well with her father. The conversation flowed easily, her parents mostly asking about Luca's travels, which he could talk about to no end. Elke was beginning to relax herself, thinking that her mother had accepted her choices in the end.

``I understand, Herr Segreti, that you are something of a photographer.'' Elke's heart skipped a beat at her mother's words. ``Even something of a famous photographer, as I gather from Elke. Do you understand at what price you are buying that fame?''

Elke saw the wistful look briefly take hold of Luca's face as he replied, ``At a most heavy one, Frau Tresler.''

Misunderstanding his meaning and feeling anger rise within her at the words, she lashed on. ``I am glad you are aware of that, for it is not you who pays it. Every bit of your fame was bought at the cost of our daughter's infamy. Are you proud of what you have done to her? Are you proud of your latest photograph?''

``More proud than of anything I've done in years.''

She rose from her chair indignantly, almost shaking with anger now. ``How dare you say that at our table after having destroyed Elke's honor? Should I cite some of the obscenities that are already written about her by your precious audience, and that will soon be said by every bum in the entire Germany?''

``Mother, stop it, please, I---''

She did not finish, Luca's hand reaching to her shoulder, his face emotionless. ``That is all right, Elke, the accusation is against me.''

``Frau Tresler, in case you haven't noticed, your daughter is a very attractive woman. I do assure you that there are twenty men thinking any one of those obscenities every time she as much as crosses the street.''

She glared at him angrily, but he carried on, ``But honor is an impregnable fortress. It can never be destroyed from without, it has to be betrayed from within. Your daughter's honor is without blemish. The bums who cannot see that are of no concern to me. Their obscenities nothing more than noises made by the mindless wind in a storm. I understand from Elke that you are a teacher, Frau Tresler. Would you refuse to deliver a lecture to a class just because some of them may be rotten? I believe you would do much better focusing on what the worthier sort of men have to say about your daughter. Then you would see that there is much reason for you to be proud of her.''

Elke saw her mother sit down without answering and return to her meal, sulking. She threw a relieved glance at Luca, thinking to herself that this could have gone much worse.

``Elke, Herr Segreti'', it was her father who broke the uncomfortable silence. ``I do not approve of what you are doing, but I respect that it is your choice. Karolyne, too, will come to respect it.'' He reached out to pat his wife.

``How can I come to respect this? Elke displayed for the whole world to see like that, likely to be discarded once her novelty has worn off and a more interesting model appears somewhere. Or are you going to tell me that you intend to marry her and live happily ever after?''

``Mother, this conversation is over, we have---''

``Elke, why do you keep interrupting? I am sure that the gentleman can speak for himself.''

``He can, mother,'' replied Elke grimly. ``And he will.''

``Ever after, Frau Tresler? As if love were something to be secured once and for all, at the wedding day, never to require having to be earned again. It is not. Love is a tribute to a vision of greatness. And it ceases to exist with that vision, any legal contracts to the contrary notwithstanding.'' Luca still appeared to her as calm as at the start of the lunch, but it was a calm with a strained edge to it, and his words came out with a faint tinge of derision. ``I will never marry your daughter, Frau Tresler. And I will have absolutely no qualms about discarding anything for something more interesting.''

``I hope that is a jest, young man.'' The voice of Elke's father was decidedly not jesting.

``Would you prefer me to stay with her instead? To keep on pretending to feel an emotion, which I no longer do, out of some sense of respect for the person who was once a source of that emotion---but is to be now but a source of resentment? And what about Elke? Sentenced to living a lie, fed counterfeit love, and wondering why she no longer sees that spark in the eyes that used to laugh for her so brightly? I am sure that Elke would hate a life of blissful ignorance in any kind of fake reality.'' He spoke the last sentence with a pointed, stern tone, but did not turn to face her. ``And someone out there, walking a lonely world, having missed a chance for joy? Three lives have just been ruined by something you would claim to be a virtue---the ugliness of sacrifice.''

He finally paused and took Elke's hand in his, looking into her eyes. There was no pain in them, but they were overflowing with confusion and uncertainty. He would have to explain further, for her sake, to finally declare to her precisely what credit he had. And let her damn him if he proves to not have enough, as she ought to have done on that night\ldots.

``You would seek to prevent this by chaining the couple together through marriage. And that alone suffices to condemn it in my eyes, although it is by far not the worst aspect of the evil that institutionalized marriage is. While on one hand, marriage destroys human joy through forced sacrifice, on the other one, it breeds smug complacency and sloth. Look how many young, passionate couples turn into apathetic spouses with beer bellies and elephant thighs. Why should they strive to be anything better?

``The wife is safe, knowing her partner cannot legally leave her anyway. And if he tried, her lawyer would skin him alive. The husband is probably happy to just have a free maid and a strumpet. For that is exactly what marriage had been for thousands of years, and its legacy remains to this day---nothing but a glorified prostitution, encouraged by the government and its trusty sidekick, the organized church---forcing women, who had been stripped of economic freedom, to sell sexual access to their bodies for subsistence. Is this what you want for Elke? She does not need me for subsistence.

``Compare this with the position of a mistress. Whenever her partner spends time with her, she knows without a shadow of a doubt that he does because he chooses so. There is no law compelling him. Out of the hundreds of women he could be with right now, or alone, he is with her. Why? Because there is no other place he'd rather be---it was her virtue and goodness that have drawn him to her. Just as she was drawn to him by similar qualities.''

He heard a soft ``Yes'' escape her lips, and turned to face her. It was impossible to misunderstand her thoughts. She, too, was thinking of that night.

He nodded in a grateful way, as to tell her that it was her who knew this long before he did; and to thank her for letting him learn it from her. Then, he turned back to her parents, and went on.

``Be it through laziness, irresponsibility or folly, suppose I changed into a man who no longer possessed the qualities that Elke loves. Or that she met someone who possessed them in larger quantity. Would you have her stay with me as a husband? I'd be the first man to tell her to run. Would it hurt me to see her go? As if my arms were cut off without anesthesia. But this is something my love for her must grant her, and I fully expect her to do the same for me.

``Mistress. You'd say it as if it were something to be ashamed of. I say it like a badge of honor. This is Elke, my mistress. The woman I've shared my life with for twenty years. We have earned it, both of us. This is Elke, my loyal wife. The woman I could not legally leave for the past twenty years. Which of them would you rather hear me say?''

He paused and looked at the table. Elke's mother was nervously crumpling a napkin, her face betraying growing exasperation and regret that she had ever invited this man into her house. Her husband sat as if listening to a speech in a foreign language which he did not understand. There was a keen attentiveness about him, however, and he watched with interest, his eyes resting more often on Elke, rather than on Luca.

``But consider at last the worst evil of marriage---the surrender to the government, or to the church, the right to regulate and interfere with that which ought to be untouchably private---what consenting adults choose to do in their bedrooms. Look for example how quick the welfare states of the West are to criminalize polygamy. Ostensibly, this is a protection of women's rights. Actually, it is their most callous destruction.

``In a polygamy, some portion of the highest quality men, those who could afford to, would choose to have multiple wives. This is good news for those men, and bad news for the remaining majority, who now have to compete for a smaller pool of wives. Many of them may even fail to win a wife for themselves altogether. Similarly, it is bad news for the most desirable women, who lose exclusive ownership of their high quality husbands, artificial as it may have been, legally enforced by a threat of jail time. But it is very good news for the vast majority of women, since they now have access to a much higher quality husband---potentially even to one of the very highest quality husbands, should they be willing to share him.

``But just as the government sends its tax collectors to take from everyone according to his ability, to then graciously redistribute to everyone according to his need---it taxes, by making illegal, the polygamous marriages of the able, to redistribute their women among the needy. Enforced monogamy is a government run slave market, where women are, like pieces of cattle, shuffled around so that every man, no matter how rotten, is pretty much guaranteed his fair share of wives.

``Now, having pushed women onto exactly the sort of men from whom they'll need protection, the government is quick to start slapping regulations on what is allowed between spouses. But if your partner is someone you wouldn't trust on his honor, how much do you think some regulation is going to restrain him? The government's meddling has created problems and more meddling has failed to solve them. Other aspects of marriage can change at any time, leaving the couple as hostages to the whims of some lousy bureaucrat. Is that what you want for Elke? Men---and women doubly so---will never be free until marriage is abolished throughout the world. We do not need the government to tell us what we mean to each other, and she doesn't need the government to protect her from me.''

He stood up from his chair and walked toward Elke. She turned sideways, rising to meet him. He took her hand and lifted it to a gentle touch of his lips.

``I'd sooner take a bullet through my head than call you my wife, Elke. Do I promise to love you and honor you all the days of my life? I do not. Any man who promises things beyond his control like that gives his word way too lightly. Do I promise to love you unconditionally? I do not. As I said to the priest, unconditional love is a horrible defraudation of virtue which is the only source of love and therefore must be its precondition. Do I promise to consider your happiness before I'll consider mine? I do not. For such a sacrifice would give you a mere illusion of happiness, soon to be dispelled, while destroying mine. But I promise I will love you for your courage and justice, for your capacity for humor and joy, for your honor, intelligence and strength, for everything that is best within you. And I will earn you by matching it with that which is best in me.''

There can be no honest trade other than of a value against a value---anything else is a disgraceful swindle, ran through Elke's mind. She was listening to him speak with her confusion disappearing fast. This was the world she had half-seen through him before, now to suddenly appear in full view for her. It made sense. And it made sense for them both to belong to it.

This resolute refusal to bend the good to the level of the worthless. The refusal to have the best within us be rewarded by anything other than the best. The refusal to see virtue dragged down and corrupted by serving vice. Is this what they mean by immaculate? I want nothing less from my life!

``You have earned me, Luca. And I will do my best to earn you in turn. The moment I default on my payment, I want you to banish me, as I'll deserve. I promise you that that is what I will do the moment you ever default on yours.'' She leaned to embrace him and kissed his mouth.

He held her kiss for a moment, then stepped away to stand behind her, wrapping his arms around her body, so that they both faced her parents.

``Frau Tresler, Herr Tresler, this is the only thing I offer, the only thing I can. And this is who we are to each other. I am asking you now, in the name of that which is virtuous and great within us, will you give us your blessings?''

Elke felt her heart beat faster and a feeling of happiness, and of pride, sweep through her body. She looked at her parents imploringly.

After a moment of silence, her father rose from the table and walked toward them with slow, heavy steps. ``Herr Segreti, you seem to be a basically good man, but I am telling you as a husband of twenty-five years that you are very, very wrong. And I am telling you as a father that you are sending my daughter, and yourself also, on a road that will bring you both only pain,'' he sighed sadly and put his hands over theirs. ``But I cannot be blind to how obviously you love Elke and she you. I give you my blessings.''

``Stefan!'' Elke's mother cried out.

``Mother, can you not see how happy I am?''

``Come, Karolyne. She is our little girl no longer, but a grown woman. It is her choice.''

There was a long, stifling silence as Elke's mother looked in turn at all the three figures standing in front of her. At last, she slowly rose from her chair, tears running from her eyes. ``This is not how I imagined it to be, Elke,'' she choked a sob. ``But I give you my blessings, too.''

Elke leapt to her and embraced her tightly. ``Thank you.''

\sectionline

They walked away from her parents' house in silence, Elke's expression betraying some new conflict raging within her. Luca could not think of any possible explanation given her manifest happiness of just a while ago. He dared not disturb the conflict, however, uneasily resolving to wait and see where it will take her.

Suddenly, she stopped walking. ``Let's sit down for a bit, Luca.'' And grabbing his hand, she pulled him to follow her to a side street leading to the nearby \begin{otherlanguage}{ngerman}Englischer Garten\end{otherlanguage}, a large park on the left bank of Isar.

She settled down in the grass, looking up at him with sad eyes. He sat down in front of her with foreboding.

``I have decided to resign from my job, Luca.''

He understood her anguished expression then, looking at her with the compassion of a man reliving a personal pain of long ago. ``I thought you loved your job, Elke.''

``I do \ldots as I would a life-long lover. I never loved him as much as I do at this very moment. But I understand now, that for my love of him, I have to beg him to run away. For I have always been delivering him, and myself alike, to bondage.''

She turned away from him and looked into the distance, not wanting him to see the tears welling in her eyes. ``But still. It hurts, Luca. It hurts so much. What have we given up? And to whom?''

``Do you mean the unearned punishment? The disguised slavery? Or the thankless effort to preserve it? We've given up nothing, Elke. It was the world who gave us up. Forget that world. It is no longer ours. It is a world where we are needed, but not one where we are wanted right now. Until the day that the world wants back again, we may not return to it.

``They think that whatever it is that they need, but can't provide, we will have to create for them. What if we simply don't? We do not have the power to change the world, but we have a tremendous power left to us, the power of choice. The choice that has always stood like a beacon---against the darkness of servitude and pain---to all the men who would not submit to it. If the world doesn't want us, we are free to choose to look for our happiness elsewhere. Do you not believe that happiness can still be ours, Elke?''

How? the way she turned toward him screamed. ``I'd love to,'' her voice sighed---then added, with whispered hopelessness, ``Do you believe it, Luca?''

``It's treason, Elke, to lose that belief. Treason to the wonders that life has to offer.'' There was no accusation in his voice, only the purity of knowledge that the guilt belonged to them both, and the serenity of certainty that there was no longer any reason for them to bear it.

``I have often doubted in the past. Until the Nürnberg tin factory. Until the moment I saw the symbol of everything I ever wished for in life---the chief-engineer who was the woman I wanted---tread, with a suggestion of innocence, grace and pride through the ruins of our world, left abandoned for the untamed nature to claim. You were the heart and the spring of that world---and its shining promise. For the moment, it was the promise, not the ruins, that mattered. I did continue to fear that I would not be worthy of that sight, and of the happiness of which it was a promise. But I never doubted its possibility again. Not once since that day.''

The self-assured calmness of his voice was like a firm hand steadying her as she stumbled under gusts of despair. She had had no idea how much that moment meant to him. I wanted you, too, then. And I knew that you felt the same way I did. The factory was truly a graveyard, only one more sinister than I'd ever suspect---the graveyard to our kind of world.

Through her tears, the spires of the Frauenkirche sparkled in the sun like a torch held high above the trees by the Statue of Liberty, calling up to her. The gusts had stopped. She would answer the call.

``I was thinking that I would open a low key, freelance agency for amateur modeling, make-up art counseling and photography. I'd have to hire a photographer somewhere though. Do you know of anyone looking for employment?''

He answered her with the greatest solemnity she had ever heard from him, ``Ay, my lady. I do.''

She threw herself into his arms and embraced him. He felt a hot trickle of tears falling onto his neck and gently rubbed her shoulders.

Then, a lone, lingering shred of denial, which she had been desperately clinging to as to a lifeline, broke inside of her. It appeared to him as if the last string holding her body erect, in defiance of a crushing weight, had been cut. She fell limply against him and began to sob without control---in protest against the monstrosity of injustice and against the world that made injustice possible. He found himself unable to speak and comfort her, stifled by a feeling of guilt for having done this to her.

He sat in the grass in silence and helplessly held her in his arms.

Gradually, her sobs grew fewer and fainter and firmness returned to her body. Tears were still rolling from her eyes, and he felt them fall on his neck. But they were the tears of a defenseless victim no more. They were the tears of release and deliverance. She slid her head smoothly against his in a caress---it had a teasing quality to it.

``Who is he? He is not too expensive, is he?''

For the second time in his life, he felt that her strength humbled his.

``He's an old miser. Will probably try to skin you alive with his rates.''

She gave a quiet chuckle, biting softly into his earlobe.

``But I think you will be able to meet those rates.''

The passion in her kiss equaled the passion in the first one they had shared.

\chapter*{Afterword}
\markboth{}{Afterword}
\addcontentsline{toc}{chapter}{Afterword}

\epigraph{If you want to make enemies, try to change something.}{Woodrow Wilson}

We hope that you have enjoyed the short story of Luca and Elke. Let us finish with a consideration of Luca's private fantasy: If enough people refused to pay the taxes, the charade would end.

It is often suggested that the correct way to effect a change in the current state of society is through patiently spreading the ideas of liberty, educating our fellow citizens and having them vote for a change via legal elections. This is an exercise in futility and wishful thinking. The leeches will always outnumber the elephants and as long as they have something worth looting to look toward, their votes will outnumber and veto any proposal for a move to a freer society. Likewise, the entrenched bureaucrats and politicians have no incentive to even consider such a change---it's their career prospects, power and prestige that are at stake.

The government is a coercive monopoly that has initiated the use of, and continues to employ, force to enslave and regulate the governed. It will take nothing less than force to counter that and change the status quo. Do not let that trouble you, however. The guilt will not be yours. A man is fully justified---no, even morally compelled---to employ retaliatory force against those who have initiated its use against him.

What kind of a force could bring down a government, however? Certainly not the force of guns---it is the force of removing your sanction. If twenty men refuse to pay their taxes, they may have a serious problem---that is, if caught. Do not make the mistake of over-estimating the efficacy of the bureaucrats, whose government jobs and cozy offices certainly do not depend on being efficient. On the other hand, if twenty million men refuse to pay their taxes, the government has a very serious problem. Ask yourself sometime, what duty do you have to relieve your oppressors of this problem.

Make no mistake about it, if you're asking what it is that you owe to the government. The answer is: nothing. And the only thing that you owe to your fellow citizens is the promise not to initiate the use of force against them. A man who, having failed to obtain what he wants through reason, reaches for a gun as for a substitute should be treated the same way like the man who reaches for a politician---like the dangerous criminal that he is.

If you are standing like a stoic pillar patiently providing for the wishes of such criminals, make sure you understand the limits of the weight you are willing to have that pillar bear. It is your life, nobody else's.

\sectionline

While Luca exhibits no reluctance to criticize the evils of the government, his views on a government-less society are kept rather vague. This is intentional, as there are a number of competing visions of such a society, and it is not the purpose of this little book to venture into political philosophy. A reader seeking an accessible account of an example of a state-less, laissez-faire society may wish to consult the excellent \emph{The Market for Liberty}, by Morris and Linda Tannehill, available for free from the Mises Institute.

\sectionline

While a number of people have helped shape the current form of the text, any logical inconsistencies or errors---both of commission and of omission---are the author's and his responsibility alone. For any comments or suggestions on the text, he may be reached at \href{mailto:Jaroslav.Tucek@gmail.com}{Jaroslav.Tucek@gmail.com}

\end{document}
